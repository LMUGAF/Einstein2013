
\chapter{Fortbewegung}

\section{Fahrrad}

Fahrradfahren lohnt sich nicht nur, weil es die schnellste und
flexibelste Möglichkeit ist, in München voranzukommen, es ist auch
gesund, schont das Klima und macht Spaß.  Es ist auch deutlich
günstiger als die häufig überfüllten öffentlichen Nahverkehrsmittel:
Drei Monate Fahrrad statt MVG und schon hat man schnell um die 100~€
gespart.

Damit kann man schon den ein oder anderen Drahtesel refinanzieren oder
hat zumindest eine Anzahlung für ein gutes, gebrauchtes Fahrrad. Diese
findet man beispielsweise bei eBay, Polizei-, Bahnhofs- und
Wohnheimsversteigerungen oder auf einem der zahlreichen Flohmärkte in
München. Kleiner Tipp: Beschränke dich bei der Suche nicht nur auf die
Stadt, sondern beziehe das Umland mit ein. Dort findet man oft
deutlich bessere Angebote.

München ist nicht nur Radlhauptstadt, sondern auch (gefühlte) Kontrollierhauptstadt. Auf ausgeschilderten Strecken Schrittgeschwindigkeit einhalten, sonst zahlt man schnell 15~€. Auf der falschen Straßenseite fahren (dies gilt auch auf der Leopold- / Ludwigstr) sowie ohne Licht bei Nacht oder Dunkelheit kosten 20~€.
Vor allem nicht unterschätzen sollte man das Rotlicht an Ampeln. Durch ignorieren dieses ist man schnell mal 100~€ los und hat zusätzlich noch Punkte in Flensburg.

\subsection*{Hier noch ein paar Tipps für den Münchner Straßenverkehr:}
\begin{itemize}
	\item Um Trambahnschienen sollte man nicht nur bei Regen und Glätte einen großen Bogen machen.
	\item Fußgänger sind in ihrem Verhalten unvorhersehbar. Die Autofahrer leider auch.
	\item Mit Helm fahren kann dein Leben retten.
	\item Um das Wiederfinden des Fahrrades zu erleichtern, sollte man es abschließen.
	
\end{itemize}

Wenn du kein eigenes Fahrrad besitzt und dir auch keins kaufen möchtest aber im Stadtkern München wohnst, sind vielleicht die Leihräder der Deutschen Bahn, namentlich Call-A-Bike \ref{callabike}, für dich interessant. Für 24,00~€ pro Jahr bekommst du als Studikon den "`Pauschaltarif"', welcher dir erlaubt jedes Fahrrad für die ersten 30 Minuten kostenfrei zu benutzen. Erst danach fällt der reguläre Preis von 0,08~€ pro Minute an. In 30 Minuten kommt man in München aber ganz schön weit mit dem Fahrrad.

Vorsicht gilt es aber bei der Rückgabe der Leihräder walten zu lassen: Ist das Fahrrad mehr als 30 Meter von der nächsten Kreuzung abgestellt kostet das 5,00~€, außerhalb des Geschäftsgebietes 10,00~€ und außerhalb der Stadtgrenzen auch schon Mal 25,00~€!

Neben Call a Bike von der Bahn gibt es auch Nextbike \ref{nextbike}, bei welchen du das Fahrrad jedoch an festen Stationen wieder abstellen musst.

\begin{urlList}
	\httpItem[Call a Bike]{callabike-interaktiv.de}{callabike}
	\httpItem[Nextbike]{nextbike.de}{nextbike}
\end{urlList}

%\begin{textblock*}{\paperwidth}(50mm,128mm)
%   \noindent\includegraphics[width={\paperwidth-50mm}]{flickr/725703538_5b9a97ecf2_b}
%	\label{img_bike}
%\end{textblock*}


\section{MVV}

\subsection*{Mit den öffentlichen Verkehrsmitteln in die Uni}\hfill\\
Die U-Bahnen U3 und U6 halten direkt am Hauptgebäude (Haltestelle Universität). Die meisten anderen Gebäude sind ebenfalls mit U-Bahn, Bus oder Tram gut zu erreichen. Genaueres zu den wichtigsten Gebäuden und naheliegenden Haltestellen finden sich auf der in der Mitte dieses Heftes zu findenden Karte.

%Informationen zu den anfallenden Kosten für den MVV (Münchner Verkehrsverbund) findest du im Kapitel "`Semesterticket und Ausbildungstarif"'.
%\subsection*{Kosten}\hfill\\
%Für die meisten Studika ist momentan der von der MVV (Münchner Verkehrsverbund) angebotene Ausbildungstarif II am interessantesten. Der Preis richtet sich dabei nach der Zahl der benötigten Zeitkartenringe, die befahren werden. Bevor du dir aber ein Ticket kaufen kannst, musst du dir ein Kundenkarte besorgen. Diese bekommst du im MVG-Kundencenter am Hauptbahnhof, Ostbahnhof oder in der Poccistr.~1--3 (alle zwischen 8:00 und 18:00~Uhr) oder online \newline http://www.mvv-muenchen.de/de/tickets-preise/tickets/schule-ausbildung-und-studium/\newline kundenkarte/index.html\#c9815

%Das Ticket gibt es mit der Gültigkeit einer Woche (9,50 -- 38,90~€) oder eines Monats (34,70 -- 142,00~€) an einem der MVG-Zeitkartenautomaten, in den MVG-Kundencentern oder den MVG-Verkaufsstellen. Monatsfahrkarten gelten bis 12Uhr des ersten Werktags des Folgemonats.\\
%~
%Wenn du in Zukunft günstiger unterwegs sein willst, kannst du bei der Initiative Ausbildungsticket, einem Bündnis aus Studika, Schülern und Azubis mitmachen:\newline \url{ausbildungsticket.de}

%Mehr Infos zum Ausbildungstarif: \url{mvg-mobil.de/tarife/ausbildungstarif.html}
%\chapter{Semesterticket und Ausbildungstarif}

\subsection{Semesterticket}

\begin{figure}[ht]
\centering
\begin{minipage}[b]{0.45\linewidth}
\includegraphics[width=\textwidth]{IsarCardSemester-MVG-ICA-Automat}
\end{minipage}
\quad
\begin{minipage}[b]{0.45\linewidth}
\includegraphics[width=\textwidth]{IsarCardSemester-MVG-ICA-Automat-Kaufbeleg}
\end{minipage}
\end{figure}

Dank des \emph{AK Mobilität zum Semesterticket München} \ref{akmobilitaet} hat München nach vielen Jahren nun auch endlich ein Semesterticket für seine Studenten. Bei der Zahlung deines Studienbeitrages ist dir sicherlich aufgefallen, dass du einen Solidarbeitrag in Höhe von 59,00~€ leisten musst. Diesen Beitrag müssen alle Stuika leisten - im Gegenzug darf damit das komplette Netz des MVV befahren werden: täglich von 18-6 Uhr, an Wochenenden und Feiertagen sogar ganztägig (daher auch "`Partyticket"' genannt).


Möchtest du dein Ticket auch außerhalb dieser Zeiten nutzen, kannst du gegen eine Zahlung von 141,00~€ die Möglichkeit an den Automaten der MVG und der Deutschen Bahn das Semesterticket zu erwerben. Im Gegensatz vom Solidarbeitrag musst du diesen Teil des Tickets aber nicht erwerben, wenn du nicht möchtest bzw. das Ticket nicht brauchst.

Für die meisten Studika, die den MVV nutzen, dürfte das Semesterticket die günstige Möglichkeit sein - es lohnt sich schon, wenn du pro Monat mehr als 23,50~€ in Fahrkarten investieren würdest.

\begin{figure}[ht]
\centering
\begin{minipage}[b]{0.45\linewidth}
\includegraphics[width=\textwidth]{IsarCardSemester-DB-Automat}
\end{minipage}
\quad
\begin{minipage}[b]{0.45\linewidth}
\includegraphics[width=\textwidth]{IsarCardSemester-DB-Automat-Kaufbeleg}
\end{minipage}
\end{figure}

Das Semesterticket - sowohl das "`Partyticket"' als auch der Teil mit Zuzahlung - sind immer für ein Semester gültig. Hier musst du dich auf die auf deinem Studienausweis aufgedruckte Laufzeit des Semesters beziehen - die anderen Hochschulen in München haben teilweise andere Laufzeiten für ihre Semester. Bitte denke auch daran, dass das Semesterticket immer nur zusammen mit deinem Studienausweis gilt, welcher wiederrum nur mit einem amtlichen Ausweisdokument gültig ist.

Wenn du beschließt ein Semesterticket am Automaten zu kaufen (halte bitte deine Matrikelnummer zur Eingabe bereit), erhälts du zwei Belege: Ein Mal das Ticket als solches und einen Zahlungsbeleg. Letzteren solltest du daheim gut aufheben, denn solltest du dein Ticket verlieren, kannst du einmalig gegen Vorlage des Zahlungsbeleges und entrichten von 5,00~€ ein zweites Semesterticket erhalten.

\begin{urlList}
	\httpItem[AK Mobilität zum Semesterticket München]{semesterticket-muenchen.de}{akmobilitaet}
\end{urlList}

\subsection{Ausbildungstarif}
Für Studika, die nur sehr selten den MVV in Anspruch nehmen, kann sich unter Umständen auch der vom MVV angebotene Ausbildungstarif II \ref{ausbildungstarif} lohnen. Der Preis richtet sich dabei nach der Zahl der benötigten Zeitkartenringe, die befahren werden. Bevor du dir aber ein Ticket kaufen kannst, musst du dir ein Kundenkarte besorgen. Diese bekommst du im MVG-Kundencenter am Hauptbahnhof, Ostbahnhof oder in der Poccistr.~1--3 (alle zwischen 8:00 und 18:00~Uhr). Alternativ kannst du deine Kundenkarte auch direkt online zum selber ausdrucken unter oder online \url{\http mvg-kundenportal.de} beantragen.

Das Ticket gibt es mit der Gültigkeit einer Woche (9,90 -- 40,40~€) oder eines Monats (36,10 -- 147,30~€) an einem der MVG-Zeitkartenautomaten, in den MVG-Kundencentern oder den MVG-Verkaufsstellen. Monatsfahrkarten gelten bis 12Uhr des ersten Werktags des Folgemonats.

\begin{urlList}
	\httpItem[Ausbildungstarif]{mvg-mobil.de/tarife/ausbildungstarif.html}{ausbildungstarif}
\end{urlList}

\section{Auto}
Du kommst im Allgemeinen mit dem Auto nicht schneller durch die Stadt, als mit dem ÖPNV oder dem Fahrrad. Spätestens bei der Parkplatzsuche vor der Uni wirst du dann merken, dass es bessere Möglichkeiten gibt, in die Uni zu kommen.
