
\section{Geld}

\subsection{Studentenwerksbeitrag}
Der Studentenwerksbeitrag setzt sich zusammen aus einem Grundbeitrag an das Studentenwerk (52,-€) und dem
Semesterticket-Sockelbeitrag (59,-€).
Diese 111,-€ müssen von allen Studika gezahlt werden, Ausnahme sind schwerbehinderte Studika, die Anspruch
auf unentgeltliche Beförderung haben.

\subsection{Jobben}
In München findest du eine Vielzahl an Nebenjobs: von Kellern oder Nachhilfe ($\geq$ 15~€/h) bis zu an der Uni selbst (ca. 8 -- 11~€/h). Deutlich höhere Stundenlöhne erhältst du, wenn du in einem der vielen IT-Unternehmen als Werkstudent arbeitest ($\geq$ 12~€/h).

Angebote findest du in Aushängen (Uni, Geschäfte) und Stadtmagazinen, aber auch in unseren Uni-Foren und unter den folgenden Adressen:
\begin{urlList}
	\httpItem{www.s-a.uni-muenchen.de/studierende/jobboerse/index.html}
	\httpItem{www.jobcafe.de}
	\httpItem{www.uni-muenchen.de/aktuelles/stellenangebote/stud_hilfskraft/index.html}
\end{urlList}

Beim Jobben solltest du den finanziellen Freibetrag der Krankenversicherung und gegebenenfalls beim BAFöG beachten, aber auch dass du unter den maximalen Wochenstunden bleibst (Krankenversicherung). Während dem Semester gelten dabei andere Grenzen als in den Semesterferien.

Dein Einkommen ist bis zu einer Grenze von ungefähr 8.000~€ (Freibetrag ohne Werbekosten usw.) steuerfrei.


\subsection{BAföG}
Im Studium kann man vom Staat finanzielle Unterstützung nach dem BundesAusbildungsförderungsGesetz erhalten. Grundsätzlich bekommen all diejenigen BAföG, die ihre Ausbildung nicht anderweitig finanzieren können (abhängig von deinem Einkommen und dem deiner Eltern / Fürsorgepflichtigen). Der Förderbetrag muss nach dem Studium zur Hälfte zurückgezahlt werden (zinsloses Darlehen), der Rest wird erlassen.

Einen ersten Eindruck von deiner Chancen auf BAföG bzw. die erwartbare
Höhe bekommst du mit dem BAföG-Rechner:
\begin{urlList}
	\httpItem{www.bafoeg-rechner.de/Rechner}
\end{urlList}

Bei einem Nein im Rechner kann es trotzdem sein, dass du BAföG
bekommst. Überlege, ob sich der Aufwand des Einreichens und
Nachreichens der Anträge für dich lohnt.

Die Bafög-Unterlagen erhältst du unter:
\begin{urlList}
	\httpItem{das-neue-bafoeg.de}
\end{urlList}
oder kannst sie online ausfüllen unter: 
\begin{urlList}
	\httpItem{bafoeg-bayern.de}
\end{urlList}

Für allgemeine Fragen kannst du dich an die allgemeine BAföG-Beratung des Studentenwerks wenden:

Helene-Mayer-Ring 9, Raum h4\\
Tel.: 089 357135-30\\
beratung-m@bafoegbayern.de\\
Mo, Di, Mi: 9:00--13:00~Uhr, 14:00--16:00~Uhr, Do: 9:00--13:00~Uhr, 14:00--17:00~Uhr, Fr: 9:00--13:00~Uhr

Konkrete Fragen besprichst du am Besten mit deinem Sachbearbeiter.

\subsection{Stipendien}
Stipendien haben meistens einen finanziellen und einen ideellen Anteil
(Seminare etc.). Für ein Stipendium ist neben passablen Noten vor
allem soziales Engagement wichtig.

\begin{urlList}
	\httpItem{www.lmu.de/deutschlandstipendium}
	\httpItem{www.lmu.de/studium/studienfinanzierung/stift}
\end{urlList}

Es gibt diverse weitere Stipendien, die nicht nur auf Noten achten.
Suchen lohnt sich!
