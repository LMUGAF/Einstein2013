\chapter{Geld}

\section{Studentenwerksbeitrag}
Der Studentenwerksbeitrag setzt sich zusammen aus einem Grundbeitrag an das Studentenwerk (52~€) und dem
Semesterticket-Sockelbeitrag (59~€).
Diese 111~€ müssen von allen Studika gezahlt werden, Ausnahmen sind schwerbehinderte Studika, die Anspruch
auf unentgeltliche Beförderung haben: Diese müssen nur den Grundbeitrag von 52,00~€ zahlen.

\section{Jobben}
In München findest du eine Vielzahl an Nebenjobs: von Kellnern oder Nachhilfe ($\geq$~15~€ pro Stunde) bis zu HiWi-Jobs und Ähnliches an der Uni selbst (ca. 8 -- 11~€ pro Stunde). Deutlich höhere Stundenlöhne erhältst du, wenn du in einem der vielen IT-Unternehmen als Werkstudent arbeitest ($\geq$~12~€ pro Stunde).

Angebote findest du in Aushängen (Uni, Geschäfte) und Stadtmagazinen, aber auch in unseren Uni-Foren \ref{infojobs} und unter den Adressen
\ref{boerse} \ref{cafe} \ref{angebot}.

Beim Jobben solltest du den finanziellen Freibetrag der Krankenversicherung und gegebenenfalls beim BAföG beachten, aber auch dass du unter den maximalen Wochenstunden bleibst (Studentenstatus). Während des Semesters gelten dabei andere Grenzen als in den Semesterferien.

Dein Einkommen ist bis zu einer Grenze von ungefähr 8.000~€ (Freibetrag ohne Werbekosten usw.) steuerfrei.

\begin{urlList}
	\httpItem{www.die-informatiker.net/forum/Jobs}{infojobs}
	\httpItem{www.s-a.uni-muenchen.de/studierende/jobboerse/index.html}{boerse}
	\httpItem{www.jobcafe.de}{cafe}
	\httpItem{www.uni-muenchen.de/aktuelles/stellenangebote/stud_hilfskraft/index.html}{angebot}
\end{urlList}


\section{BAföG}
Im Studium kann man vom Staat finanzielle Unterstützung nach dem BundesAusbildungsförderungsGesetz erhalten. Grundsätzlich bekommen all diejenigen BAföG, die ihre Ausbildung nicht anderweitig finanzieren können (abhängig von deinem Einkommen und dem deiner Eltern / Fürsorgepflichtigen). Der Förderbetrag muss nach dem Studium zur Hälfte zurückgezahlt werden (zinsloses Darlehen), der Rest wird erlassen.

Einen ersten Eindruck von deineb Chancen auf BAföG bzw. von der erwartbaren
Höhe bekommst du mit dem BAföG-Rechner \ref{rechner}.

Bei einem 'Nein' im Rechner kann es trotzdem sein, dass du BAföG
bekommst. Überlege, ob sich der Aufwand des Einreichens und
Nachreichens der Anträge für dich lohnt.

Für allgemeine Fragen kannst du dich an die allgemeine BAföG-Beratung des Studentenwerks wenden:

Helene-Mayer-Ring 9, Raum h4\\
Tel.: 089 357135-30\\
\mail{beratung-m@bafoegbayern.de}\\
Mo, Di, Mi: 9:00--13:00~Uhr, 14:00--16:00~Uhr, Do: 9:00--13:00~Uhr, 14:00--17:00~Uhr, Fr: 9:00--13:00~Uhr

Konkrete Fragen besprichst du am Besten mit deinem Sachbearbeiter.

\begin{urlList}
	\httpItem[BAföG-Rechner]{www.bafoeg-rechner.de/Rechner}{rechner}
	\httpItem[BAföG-Unterlagen]{das-neue-bafoeg.de}
	\httpItem[Unterlagen online ausfüllen]{bafoeg-bayern.de}
\end{urlList}

\section{Stipendien}
Stipendien haben meistens einen finanziellen und einen ideellen Anteil
(Seminare etc.). Für ein Stipendium ist neben passablen Noten vor
allem soziales Engagement wichtig.

Es gibt diverse weitere Stipendien, die nicht nur auf Noten achten.
Suchen lohnt sich!

\begin{urlList}
	\httpItem{www.lmu.de/deutschlandstipendium}
	\httpItem{www.lmu.de/studium/studienfinanzierung/stift}
\end{urlList}

\section{Krankenversicherung}

Für Studika an staatlichen und staatlich anerkannten Hochschulen
besteht generell eine Kranken- und Pflegeversicherungspflicht. Diese
Pflicht beginnt mit der Immatrikulation an der Hochschule und endet in
der Regel nach Ablauf des 14. absolvierten Fachsemesters oder mit der
Vollendung des 30. Lebensjahres. Dabei sind verschiedene Formen der
Versicherung möglich.

\subsection*{Ich bin Studikon und meine Eltern sind gesetzlich krankenversichert}

Bis zum Ende des 25. Lebensjahres sind alle Studienanfangenden, 
deren Einkommen unter 385~€ im Monat liegt, über die Eltern in der
gesetzlichen Familienversicherung mitversichert.
Bei einem Minijob ist ein Einkommen von bis zu 450~€ monatlich erlaubt. BAföG und Unterhaltszahlungen
der Eltern gelten nicht als Einkommen. Für diejenigen, die
zwischen Abitur und Studienbeginn Zivil- oder Wehrdienst abgeleistet
haben, verlängert sich die Zeit in der Familienversicherung um die
Dauer des Dienstes. Während dieser Zeit muss man sich generell um
nichts kümmern – von der Bescheinigung über die Versicherung für die
Immatrikulation einmal abgesehen. Es besteht die Möglichkeit, sich
während der ersten drei Monate des Studiums nach Studienbeginn von der
gesetzlichen Krankenversicherung befreien zu lassen und sich privat zu
versichern.

\subsection*{Ich bin Studikon und meine Eltern sind privat krankenversichert}

Bei einer privaten Krankenversicherung gibt es die Möglichkeit, bis zu
dem 25. Lebensjahr im Rahmen einer Familienversicherung
mitversichert zu sein. In der privaten Krankenversicherung muss
allerdings pro Familienmitglied ein bestimmter Betrag gezahlt
werden. Die kostenlose Familienversicherung gibt es nur in der
Gesetzlichen. Wenn man sich privat familienversichern will, muss man
sich innerhalb der ersten drei Monate des Studiums von der
studentischen Pflichtversicherung befreien lassen, was aber generell
kein all zu großer Aufwand ist. Man kann sich auch dafür entscheiden,
sich schon zu Studienbeginn gesetzlich über die studentische
Krankenversicherung zu versichern und damit aus der privaten
Familienversicherung auszutreten.

\subsection*{Ich bin Studikon und verdiene mehr als 385~€ im Monat}

Wer als Studikon mehr als 385~€ im Monat verdient, fällt aus der
Familienversicherung heraus. Eine Ausnahme stellen nur so genannte
Minijobs dar, bei denen die Verdienstgrenze auf 450~€ angehoben
wird. Wer diese Grenze nur sehr knapp überschreitet, sollte vor
Studienbeginn noch einmal genau nachrechnen: Bei der Rechnung zur
Krankenversicherung kann die so genannte Werbungskostenpauschale
geltend gemacht werden, mit der man noch einmal etwas mehr als 70~€
vom Monatsgehalt abziehen kann. Wer immer noch oberhalb der
Einkommensgrenze liegt, kann zwischen einer gesetzlichen und privaten
Krankenversicherung wählen. Dies will wohl überlegt sein, denn die Befreiung
ist nur innerhalb der ersten drei Monate nach der Einschreibung möglich und
unwiderruflich. Auch nach dem Studium kann man nicht ohne Weiteres zur gesetzlichen
Krankenversicherung wechseln.

\subsection*{Ich bin Studikon und über 25}

Sofern sich die Altersgrenze nicht durch das Ableisten eines Wehr-
oder Zivildienstes nach hinten verschoben hat, endet mit dem
26. Geburtstag die Mitgliedschaft in der Familienversicherung. 
Studika mit einer privaten Versicherung sollten sich bei ihrer
Krankenkasse nach besonderen Studententarifen erkundigen. Gesetzlich
Versicherte können sich über die studentische
Krankenversicherung versichern lassen. Nach Vollendung des 30.~Lebensjahres
oder nach dem 14.~Fachsemester endet die Versicherungspflicht und damit
auch der günstige Tarif. Sofern nicht durch einen Job die Versucherungspflicht
als Arbeitnehmer in Kraft tritt, sollte man sich freiwillig
weiter versichern.

\subsection*{Ich bin Studikon und über 30}

Mit dem 31. Geburtstag erlischt auch die Möglichkeit der
Mitgliedschaft in der studentischen gesetzlichen
Krankenversicherung. Nach Ablauf dieser Frist kann man sich 
als freiwilliges Mitglied in der Krankenversicherung einstufen lassen. 
Bis maximal 6~Monate gibt es einen vergünstigten Übergangstarif für
Studika, danach bleibt nur die normale freiwillige Versicherung.
Dies ist aber nur möglich, wenn
man vorher auch gesetzlich versichert war. Bei der privaten
studentischen Krankenversicherung liegt die Altersobergrenze bei 34
Jahren. Allerdings gibt es hier anschließend nicht die Möglichkeit den
Freiwilligenstatus einzunehmen.

\subsection*{Studentische Versicherung in der gesetzlichen Krankenkasse}

Die Beitragssätze sind bei allen Krankenkassen gleich (ca. 80~€ im
Monat). Unterschiede gibt es bei den einzelnen Kassen aber oft im
Leistungsumfang, so dass ein Vergleich lohnenswert ist. Da die Krankenkassen bei
Studika auf künftig gut verdienende Mitglieder hoffen, werden
diese an den Hochschulen manchmal regelrecht umworben. Dementsprechend
sind die angebotenen Leistungen der Krankenkassen meist nicht
schlecht.

\subsection*{Studentische Versicherung in der privaten Krankenkasse}

Natürlich sind auch die privaten Versicherer daran interessiert,
Mitglieder aus dem akademischen Bereich zu generieren. Und so sind die
Konditionen der privaten studentischen Versicherungen ebenfalls recht gut. Zudem ist der
Leistungsumfang üblicherweise wesentlich höher. Vor allem für
männliche Studenten kann sich der Abschluss einer privaten
studentischen Krankenversicherung finanziell durchaus lohnen. Für weibliche
Versicherungspflichtige sind die Tarife der
studentischen Krankenversicherung meist höher, können aber unter
bestimmten Umständen trotzdem attraktiv sein. Das kann etwa der Fall
sein, wenn das Einkommen hoch genug ist um in der gesetzlichen
Krankenkasse zusätzlich beitragspflichtig zu sein.

Die private studentische Krankenversicherung hat aber auch Nachteile. So ist es nach
Abschluss des Studiums meist schwierig und unter Umständen auch gar
nicht möglich, wieder in eine gesetzliche Krankenkasse zu
wechseln. Nötig ist dazu das dauerhafte Absinken des Einkommens unter
die Versicherungspflichtgrenze (knapp 50.000~€). Bei den Privaten sind
die Tarife meist in jungen Jahren günstig und steigen später deutlich
an. Wer später die günstigeren Tarife der Gesetzlichen nutzen möchte,
soll sich dem Solidarprinzip hier nicht im Vorfeld entziehen. Dies ist
zumindest der Standpunkt der Gesetzgeber.

\footnotesize (zusammengestellt aus {\url{studenten-krankenversicherung.net}})
\normalsize
