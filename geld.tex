
\section{Geld}

\subsection{Studienbeitrag}
Der Studienbeitrag setzt sich zusammen aus einem Beitrag an das Studentenwerk (42~€) und Studiengebühren (500~€).

\textbf{Frist für die Zahlung: 16. November 2012}

Von den Studiengebühren kann man sich befreien lassen mit einem der
folgenden Gründe:
\begin{itemize}
        \item Man hat mindestens zwei Geschwister, für die deine Familie
          Kindergeld bezieht
	\item Man hat einen Geschwisterteil, der Studiengebühren/-beiträge an einer Hochschule in der EU zahlt
	\item Man erzieht ein Kind
	\item Man ist chronisch krank
	\item Die Eltern haben extrem wenig Geld (unzumutbarer Härtefall)
	\item etc.
\end{itemize}

Die genauen Beschreibungen findet ihr unter: \url{www.lmu.de/studium/administratives/gebuehr/studiengebuehren/befreiungen/befreiungsgruende}

\textbf{Frist für die Befreiung: 31. Oktober 2012}

Wenn du zu den besten 10\% deines Jahrganges zählst, können dir die
Studiengebühren rückwirkend erlassen werden.

Sachgebiet 5 -- Studienbeiträge, HGB, E114\\
Mo, Di, Mi, Fr: 08:30 -- 11:30~Uhr, Do: 13:30 -- 15:00~Uhr\\
\url{www.lmu.de/studium/administratives/gebuehr/studiengebuehren}


\subsection{Jobben}
In München findest du eine Vielzahl an Nebenjobs: vom Café, Nachhilfe ($\geq$ 15~€/h) bis an der Uni selbst (ca. 9 -- 11~€/h). Deutlich höhere Stundenlöhne erhälst du, wenn du in einem der vielen IT-Unternehmen als Werkstudent arbeitest ($\geq$ 12~€/h).

Angebote findest du in Aushängen (Uni, Geschäfte) und Stadtmagazinen, aber auch unter den folgenden Adressen:
\begin{itemize}

\item in den Foren die-informatiker.net, die-physiker.org und die-mathematiker.net

\item \url{www.jobcafe.de}

\item \url{www.s-a.uni-muenchen.de/studierende/jobboerse/index.html}

\item \url{www.uni-muenchen.de/aktuelles/stellenangebote/stud_hilfskraft/index.html}
\end{itemize}

Beim Jobben solltest du den finanziellen Freibetrag der Krankenversicherung und ggf. beim BAFöG beachten, aber auch dass du unter den maximalen Wochenstunden bleibst (Krankenversicherung). Im Semester gelten dabei andere Grenzen als in den Semesterferien.

Dein Einkommen ist bis zu einer Grenze von ungefähr 8.000~€ (Freibetrag ohne Werbekosten etc.) steuerfrei.


\subsection{BAföG}
Im Studium kann man vom Staat finanzielle Unterstützung nach dem BundesAusbildungsförderungsGesetz erhalten. Grundsätzlich bekommen all diejenigen BAföG, die ihre Ausbildung nicht anderweitig finanzieren können (abhängig von deinem Einkommen und dem deiner Eltern / Fürsorgepflichtigen). Der Förderbetrag muss nach dem Studium zur Hälfte zurückgezahlt werden (zinsloses Darlehen), der Rest wird erlassen.

Einen ersten Eindruck von deiner Chancen auf BAföG bzw. die erwartbare
Höhe bekommst du mit dem BAföG-Rechner:
\url{www.bafoeg-rechner.de/Rechner}

Bei einem Nein im Rechner kann es trotzdem sein, dass du BAföG
bekommst. Überlege, ob sich der Aufwand des Einreichens und
Nachreichens der Anträge für dich lohnt.

Die Bafög-Unterlagen erhältst du unter: \url{das-neue-bafoeg.de}\\
oder kannst sie online ausfüllen unter: \url{bafoeg-bayern.de}

Für allgemeine Fragen kannst du dich an die allgemeine BAföG-Beratung des Studentenwerks wenden:

Helene-Mayer-Ring 9, Raum h4\\
Tel.: 089 357135-30\\
beratung-m@bafoegbayern.de\\
Mo, Di, Mi: 9:00--13:00~Uhr, 14:00--16:00~Uhr, Do: 9:00--13:00~Uhr, 14:00--17:00~Uhr, Fr: 9:00--13:00~Uhr

Konkrete Fragen besprichst du am Besten mit deinem Sachbearbeiter.

\subsection{Stipendien}
Stipendien haben meistens einen finanziellen und einen ideellen Anteil
(Seminare etc.). Für ein Stipendium ist neben passablen Noten vor
allem soziales Engagement wichtig.

\begin{itemize}
	\item Deutschland-Stipendium: \url{www.lmu.de/deutschlandstipendium}
	\item Sonstige Stipendien: \url{www.lmu.de/studium/studienfinanzierung/stift}
\end{itemize}

Es gibt diverse weitere Stipendien, die nicht nur auf Noten achten.
Suchen lohnt sich!
