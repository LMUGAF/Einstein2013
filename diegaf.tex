\section{�ber die GAF}

\subsection{\ldots{} dieses Heft}

Bei der ganzen Informationsflut, die in der ersten Unizeit auf dich
einst�rzt, hoffen wir dir mit unserem \emph{Ersti-Einstein} einen
kleinen Ratgeber an die Hand zu geben.  Der Ersti-Einstein b�ndelt
Wichtiges, erkl�rt dir Nichtoffensichtliches, und versucht bei vielen
Problemen zumindest erste L�sungsans�tze zu bieten.

Da wir nicht mehr alle Probleme, die ein Ersti hat, nachvollziehen
k�nnen, und jedes Jahr neue Probleme gefunden werden, w�rden wir uns
freuen, wenn du uns fehlende Informationen unter
\mail{einstein@fs.lmu.de} mitteilst.


\subsection{weg damit!}

Wir sind die GAF -- Gruppe Aktiver Fachschaftika -- ein
Zusammenschluss der Fachschaften der Fachbereiche
Mathematik,
Physik
und Informatik (also auch Wirtschaftsmathe, Medieninfo und
Meteorologie), sowie den dazugeh�rigen Lehramtsstudieng�ngen.

\paragraph{Was ist eine Fachschaft?}

Eine Fachschaft sind alle Studenten in einem Studiengang, das hei�t die Fachschaft Physik besteht zum Beispiel aus \emph{allen} Studenten (ob ihr wollt oder nicht), die in Physik eingeschrieben sind.

Wenn man von \emph{der Fachschaft} spricht, meint man normalerweise die Aktiven,
zum Beispiel die Organisatoren der O-Phase.

\paragraph{Was macht die Fachschaft?}
\begin{itemize}
\item Repr�sentation studentischer Interessen auf allen Ebenen, d.h. in Universit�tsgremien z.B. dem Fakult�tsrat, der Studiengeb�hrenkommission oder Berufungskommissionen f�r neue Professoren \ldots
\item Verwaltung alter Klausuren, Pr�fungsprotokolle und allem Anderen, was beim Studium hilft.
\item Information (O-Phase) und Beratung. Wenn du Fragen hast und nicht mal wei�t, wen du fragen sollst, frag' uns.
\item Bespa�ung: Wir organisieren Partys und andere Aktionen (Lange Nacht der Uni, Fakult�tsfest, \ldots), damit auch der soziale Aspekt an der Uni nicht zu kurz kommt.
\item Anlaufstelle f�r Fragen aller Art; Sammeln von Meinungen f�r die Fakult�t und Informationen aus der Fakult�t, die wir an dich weiter leiten.
\end{itemize}

\paragraph{Wie machen die das?}
Die Fachschaft an sich bekommt Geld, das aber nur zum Nutzen der
Studenten eingesetzt werden darf, d.h. diejenigen, die in der
Fachschaft aktiv sind, tun dies ehrenamtlich und unentgeltlich.

\clearpage

\paragraph{Ich will mitmachen! Wie?}\label{mitmachen}\hfill\\

Komm einfach vorbei:
\begin{itemize}
	\item ins Fachschaftszimmer (Theresienstr., B037)
	\item zur Fachschaftssitzung (Termine unter \url{gaf.fs.lmu.de/wir/fachschaftssitzung})
\end{itemize}
oder sprich uns bei der O-Phase an.


\paragraph{Wie kann ich euch erreichen?}\label{gafKontakt}\hfill\\[1em]
%XXX To do: seperate Box/ optisch aufwerten
\begin{tabular}{ l l l l }
Telefon&089 / 2180\emd{}4382\\
Telefax&089 / 2180\emd{}4391\\
&\\
&\mail{gaf@fs.lmu.de}\\
&\mail{einstein@fs.lmu.de}\\
&\mail{gumbel@fs.lmu.de}\\
&\\
&\url{gaf.fs.lmu.de}\\
&\url{facebook.com/gaflmu}\\
&\\
IRC & \url{#gaf} auf freenode
\end{tabular}

\paragraph{Andere Fachschaften}
\begin{itemize}
	\item \textbf{Medieninformatik:} \url{mi.fs.lmu.de} und \url{facebook.com/FS.Medieninformatik.LMU}
	\item \textbf{Bioinformatik:} \url{www.bioinformatik-muenchen.com/bioinfocom/fachschaft/}
	\item \textbf{Meteorologie:} \url{www.meteo.physik.lmu.de/~studmet}
\end{itemize}

\skiptobottom
%XXX To do: neues Bild
\centerline{\includegraphics[width=0.8\textwidth]{aktive-fachschaft_print}}
