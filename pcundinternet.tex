\section{Computer und Internet}

To Do: Introduction
Hier erfahrt ihr, welche Zugangskennungen etc. f�r euch ins Uni-Netz relevant sind bla bla.

\subsection{Online-Dienste der LMU}

\paragraph{Online-Selbstbedienungsfunktionen}\hfill\\
\url{www.lmu.de/studium/studium_aktuell/neuigkeiten/studkanz/system.html}
\begin{itemize}
	\item Bescheinigungen: Immatrikulation, Studienverlauf, gezahlte Beitr�ge
	\item Adress-/ Telefonnummern�nderung
	\item Formular zur Pr�fungsanmeldung
\end{itemize}

\paragraph{CAMPUS LMU}\hfill\\
\url{campus.lmu.de} oder \url{www.portal.lmu.de}
\begin{itemize}
	\item Aktivierung der Campuskennung
	\item Zugang zum E-Mailaccount und Benutzerkonto (An-/Abmeldung von Newslettern der LMU)
	\item Zugang zum LSF (Vorlesungsverzeichnis)
	\item Personalisierbare Startseite mit Feldern wie Veranstaltungen, Mensa, Quicklinks etc.
\end{itemize}

\paragraph{Vorlesungsverzeichnis}\hfill\\
\url{www.lsf.lmu.de}
\begin{itemize}
	\item �bersicht �ber alle Veranstaltungen der LMU
	\item Stundenplan-Tool (etwas merkw�rdig zu bedienen)
	\item Anmeldung zu Kursen und Klausuren (BWL, VWL)
        \item Notenauszug, nicht immer aktuell (Physik)
\end{itemize}

\paragraph{UniWorX (Informatik, Medieninformatik)}\hfill\\
\url{uniworx.ifi.lmu.de}
\begin{itemize}
	\item Anmeldung zu Kursen und Klausuren im Bereich Informatik
	\item Abgabe von �bungsbl�ttern
\end{itemize}

\paragraph{Pr�fungsverwaltungs- und Informationssystem (PVI)}\hfill\\
\url{pvineu.ifi.lmu.de}
\begin{itemize}
	\item Notenabfrage f�r Informatik und Medieninformatik
\end{itemize}

\clearpage

\subsection{E-Mail}
Damit du nicht unterfordert wirst, besitzt du direkt von Anfang an zwei verschiedene E-Mailadressen. Bei beiden E-Mailadressen ist es m�glich eine Weiterleitung einzurichten.\\

Die erste Adresse besitzt jeder Student der LMU, w�hrend die Zweite f�r alle Nutzer der CIP-Pools ist.

\paragraph{F�r alle LMU-Studierende}
\begin{itemize}
	\item <vorname.nachname>@campus.lmu.de (bzw. was ihr angegeben habt)
	\item Webmail unter: \url{mailbox.portal.uni-muenchen.de}
\end{itemize}

\paragraph{Informatik und MedienInformatik}\hfill\\
Sollte unbedingt weitergeleitet oder abgerufen werden, da hier�ber der Gro�teil des Informatik-Mailverkehrs abl�uft. Insbesondere das Abgabesystem Uniworx schickt euch Informationen der Dozenten und Korrekturergebnisse an diese Adresse.
\begin{itemize}
	\item <accountname>@cip.ifi.lmu.de
	\item Webmail unter: \url{webmail.ifi.lmu.de}
	\item Infos unter: \url{www.rz.ifi.lmu.de/Dienste/Mailsystem.html}
	\item Beantragung der Kennung: \url{www.rz.ifi.lmu.de}
\end{itemize}

\paragraph{Physik und Meteorologie}\hfill\\
An diese Adresse werden Ank�ndigungen des Pr�fungsamtes und
Physik-Newsletter gesendet.

\begin{itemize}
	\item <vorname.nachname>@physik.uni-muenchen.de
	\item Webmail unter \url{webmail.physik.uni-muenchen.de}
	\item Infos unter \url{it.physik.uni-muenchen.de/dienste/kommunikation/e-mail}
        \item Passwort wie bei \url{campus.lmu.de}.
\end{itemize}

\paragraph{Mathematik und Wirtschaftsmathematik}
\begin{itemize}
	\item <seltsameKombination>@math.lmu.de
	\item Infos nur aus dem CIP-Pool abrufbar, Einf�hrung/ Beantragung E-Mailadresse bei Herrn Spann (Anmeldung in der Theresienstr., B124)
	\item Weiterleitung �ber Shell-Kommando \verb|echo "neue Adresse" >~/.forward|
\end{itemize}

\subsection{CIP-Pools}
Computer, soziale Kontake und Drucken (600 Seiten/Semester kostenlos).\\

\begin{tabular}{l p{10cm}}
\textbf{Mathematik}	&	Theresienstr., BU135 und BU136, \newline Wendeltreppe nach unten\\

\textbf{Physik, Meteorologie}	&	Schellingstr. 4 Erdgeschoss, H037 und H022\\
\textbf{Medieninformatik, Informatik}	&	Oettingenstr. 67, BU102, LU112, LU114 und LU117 (Keller und Barracken)\\
\textbf{Medieninformatik zus�tzlich}	&	Amalienstr. 17, EG\\
\textbf{F�r alle*}	&	Theresienstr., 1. Stock B115\newline
\footnotesize{$^*$Physik: arbeiten, nicht drucken; \newline Informatik: drucken, nicht arbeiten}

\end{tabular}


\subsection{Internet: WLAN, VPN und Eduroam}

Um mit deinem Laptop in der Uni ins Internet zu gehen, brauchst du
deinen Campus-Account. Damit lassen sich die WLAN-Services des
Leibniz-Rechen\-zentrums (LRZ) nutzen.

\paragraph{Eduroam}
Wir empfehlen dir, das WLAN mit dem Namen (SSID) \emph{eduroam}, auf deinen Ger�ten einzurichten. Mit diesem einmal eingerichteten Eduroam kannst du weltweit an vielen Universit�ten und Forschungsinstituten automatisch das dortige WLAN nutzen. Unter \url{lrz.de/services/netz/mobil/eduroam} findest du ausf�hrliche Anleitungen f�r die meisten Betriebssysteme und Smartphones.
(Die ben�tigte LRZ-Kennung findest du in deinem Campus-Account unter 'Benutzerkonto' $\rightarrow$ 'E-Mail-Einstellungen'.)

Die Anleitung \emph{Windows (alt)} bezieht sich �brigens einfach auf
Windowsversionen vor Windows 7.

Falls du nun in der Uni sitzt und dich fragst, wie du ohne Internet
die Anleitung durchlesen oder deine LRZ-Kennung herausfinden sollst:

\paragraph{LRZ}

Au�er eduroam gibt es noch die M�glichkeit, das Netz mit der SSID
\emph{lrz} zu verwenden. \emph{lrz} ist zun�chst ein unverschl�sseltes
Netzwerk, das nur den Zugriff auf die Website des
Leibnitz-Rechenzentrums gestattet. Von dort kannst du dir dort die
vorkonfigurierte Clientsoftware AnyConnect herunterladen, welche dich
nach Anmelden mit deiner Campuskennung in ein VPN (Virtual Private
Network) des LRZ einbucht. Aus Netzwerksicht verh�lt sich dein Rechner
dann wie alle anderen Rechner im M�nchener Wissenschaftsnetz. So
kannst du nicht nur normal surfen, sondern auch von au�en auf das
M�nchner Wissenschaftsnetz zugreifen und zum Beispiel bestimmte
Artikel aus der Bibliothek lesen.

Die Clientsoftware ist �brigens au�erhalb der Uni praktisch, um deine
HTTP-Verbindungen zu verschl�sseln, etwa wenn du dich in einem
ungesch�tzen WLAN befindest.

\paragraph{Microsoft DreamSpark}
Studenten der Physik und Informatik (auch im Nebenfach) bekommen �ber
Microsoft DreamSpark (fr�her MSDNAA) viele Microsoftproduktlizenzen
gratis, darunter Windows, Visual Studio und viele
Microsoft Office-Komponenten, jedoch \textbf{nicht} Word, Excel und PowerPoint.

\paragraph{Proxyeinrichtung}

Um E-Books, Paper, und wissenschaftliche Journale der Uni-Bib
herunterladen zu k�nnen, ben�tigst du einen Proxyzugang. Dazu musst du
entweder im MWN (M�nchner Wissenschaftsnetz, z.B. im CIP) sitzen oder
mittels dem VPN-Client so wirken als ob und dann einen Proxy in deinem
Browser eintragen. Wie man das macht wird hier erkl�rt:
\url{www.ub.uni-muenchen.de/?id=2402}.

Einfacher gehts seit Neustem mit dem Easyproxy:\\
\url{www.ub.uni-muenchen.de/elektronische-medien/hilfe-anleitungen/easyproxy/}
