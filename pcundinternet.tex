\chapter{Computer und Internet}
Hier erfährst du, welche Möglichkeiten du hast die CIP-Räume der Uni zu nutzen (je nach
Fach verschieden notwendig, hilfreich und ausgestattet), wie du Zugang zum Uni-WLAN erhältst 
und welche anderen nützlichen Dinge die Uni sonst noch anbietet.

\section{Online-Dienste der LMU}
\subsection*{Online-Selbstbedienungsfunktionen}
\begin{itemize}
	\item Bescheinigungen für Immatrikulation, Studienverlauf, gezahlte Beiträge
	\item Änderung von Adress-/Telefonnummern
	\item Formular zur Prüfungsanmeldung
\end{itemize}
\begin{urlList}
	\httpItem{www.lmu.de/studium/studium_aktuell/neuigkeiten/studkanz/system.html}
\end{urlList}

\subsection*{Campus LMU}
\begin{itemize}
	\item Aktivierung der Campuskennung
	\item Zugang zum E-Mail-Account
	\item Zugang zum Benutzerkonto (An-/Abmeldung von Newslettern der LMU)
	\item Zugang zum LSF (Vorlesungsverzeichnis)
\end{itemize}
\begin{urlList}
	\httpItem{www.campus.lmu.de} 
	\httpItem{www.portal.lmu.de}
\end{urlList}

\subsection*{Vorlesungsverzeichnis (Lehre Studium Forschung -- LSF)}
\begin{itemize}
	\item Übersicht über (fast) alle Veranstaltungen der LMU
	\item Stundenplan-Tool (etwas merkwürdig zu bedienen)
	\item Anmeldung zu Kursen und Klausuren (BWL, VWL)
	\item Notenauszug, nicht immer aktuell (Physik)
\end{itemize}
\begin{urlList}
	\httpItem{www.lsf.lmu.de}
\end{urlList}

\subsection*{UniWorX\subjectList{\subjectI{}\subjectMI{}}}
\begin{itemize}
	\item Anmeldung zu Kursen und Klausuren
	\item Abgabe von Übungsblättern
	\item Mit Campus- oder Cip-Kennung nutzbar
\end{itemize}
\begin{urlList}
	\httpItem{www.uniworx.ifi.lmu.de}
\end{urlList}

\subsection*{Prüfungsverwaltungs- und Informationssystem (PVI)\subjectList{\subjectI{}\subjectMI{}}}
\begin{itemize}
	\item Notenauszug, verbuchte Prüfungen
\end{itemize}
\begin{urlList}
	\httpItem{pvineu.ifi.lmu.de}
\end{urlList}

\section{E-Mail}
Damit du nicht unterfordert wirst, besitzt du direkt von Anfang an zwei verschiedene E-Mail-Adressen. Bei beiden E-Mail-Adressen ist es möglich und auch wärmstens empfohlen eine Weiterleitung einzurichten.

Die Campus-Adresse besitzt jedes Studikon der LMU, während die CIP-Adresse für die Nutzer der CIP-Pools ist.

\subsection*{Für alle Studika der LMU}
\begin{itemize}
	\item <vorname.nachname>@campus.lmu.de (bzw. was ihr angegeben habt)
	\item Zum Weiterleiten einfach unter \ref{webmail} links unten auf Weiterleitung klicken und eine andere E-Mail-Adresse angeben.
\end{itemize}
\begin{urlList}
	\httpsItem[Webmail]{mailbox.portal.uni-muenchen.de}{webmail}
\end{urlList}

\subsection*{Informatik und Medieninformatik \subjectList{\subjectI\subjectMI}}
Sollte unbedingt abgerufen oder weitergeleitet werden, da hierüber der Großteil des Informatik-Mailverkehrs abläuft. 
\begin{itemize}
	\item <accountname>@cip.ifi.lmu.de
	\item Beantragung der Kennung während der O"~Phase, in den ersten zwei Wochen des Semesters oder zu den Sprechstunden der RBG 
		(Mo--Fr, 13--17 Uhr) jeweils in der Oettingenstr. 67, in LU113.
\end{itemize}
\begin{urlList}
		\httpsItem[Webmail]{webmail.ifi.lmu.de}
		\httpItem[Infos]{www.rz.ifi.lmu.de/Dienste/Mailsystem.html}
\end{urlList}

\subsection*{Physik und Meteorologie\subjectList{\subjectP}}
An diese Adresse werden Ankündigungen des Prüfungsamtes und
Physik-Newsletter gesendet.
\begin{itemize}
	\item <vorname.nachname>@physik.uni-muenchen.de
	\item Passwort ist das gleiche wie bei der Campus-Adresse
\end{itemize}
\begin{urlList}
	\httpItem[Webmail]{webmail.physik.uni-muenchen.de}
	\httpItem[Infos]{www.it.physik.uni-muenchen.de/dienste/kommunikation/e-mail}
\end{urlList}

\subsection*{Mathematik und Wirtschaftsmathematik\subjectList{\subjectM\subjectW}}
\begin{itemize}
	\item <seltsameKombination>@math.lmu.de
	\item Account bei Herrn Spann (Theresienstr.37--41, B124) beantragen
	\item Weiterleitung über Shell-Kommando \verb|echo "neue Adresse" >~/.forward|
\end{itemize}

\section{CIP-Pools}
In CIP-Pools\footnote{Computer-Investitions-Programm} findest du Rechnerarbeitsplätze und Drucker, sowie teils Scanner. Das Druckerkontingent beträgt für Mathematika, Physika und Statistika 500 Seiten pro Semester. Informatika haben 600 Seiten pro Semester zur freien Verfügung. Einige CIP-Pools haben auch Farbdrucker, deren Kontingent ist kleiner.

\begin{tabularx}{\linewidth}{lX}
\textbf{Mathematik}                   & Theresienstr. 37--41, BU135 und BU136, Wendeltreppe nach unten\\
\textbf{Physik, Meteorologie}         & Schellingstr. 4 Erdgeschoss, H037 und H022\\
\textbf{Medieninformatik, Informatik} & Oettingenstr. 67, BU102, LU112, LU114 und LU117 (Keller und Barracken)\\
\textbf{Medieninformatik zusätzlich}  & Amalienstr. 17, EG\\
\textbf{Für alle*}                    & Theresienstr. 37--41, 1. Stock B115 \newline
\footnotesize{$^*$Physik: arbeiten, nicht drucken; \newline $\phantom{^*}$Informatik: drucken, nicht arbeiten}
\end{tabularx}

\section{Internet: WLAN, VPN und Eduroam}
Um mit deinem Laptop in der Uni ins Internet zu gehen, brauchst du
deine Campus-Kennung. Damit lassen sich die WLAN-Services des
Leibniz-Rechen\-zentrums (LRZ) nutzen.

\subsection*{Eduroam}
Wir empfehlen dir, das WLAN mit dem Namen (SSID) \emph{eduroam}, auf deinen Geräten einzurichten. Mit diesem einmal eingerichteten Eduroam kannst du weltweit an vielen Universitäten und Forschungsinstituten automatisch das dortige WLAN nutzen. Unter \ref{eduroam} findest du ausführliche Anleitungen für die meisten Betriebssysteme und Smartphones.
Diese Seite ist über das \emph{lrz} Wi-Fi erreichbar.
(Die benötigte LRZ-Kennung findest du in deinem Campus-Account unter `Benutzerkonto' $\rightarrow$ `E-Mail-Einstellungen'.)

%TODO noch ein Hinweis auf das weniger gefüllte eduroam-a?
Falls du nun in der Uni sitzt und dich fragst, wie du ohne Internet
die Anleitung durchlesen oder deine LRZ-Kennung herausfinden sollst, 
findest du die Antwort im Abschnitt LRZ.
\begin{urlList}
	\httpItem{www.lrz.de/services/netz/mobil/eduroam}{eduroam}
\end{urlList}

\subsection*{LRZ}
Außer eduroam gibt es noch die Möglichkeit, das Netz mit der SSID
\emph{lrz} zu verwenden. \emph{lrz} ist zunächst ein unverschlüsseltes
Netzwerk, das nur den Zugriff auf die Website des
Leibnitz-Rechenzentrums gestattet. Von dort kannst du dir entweder die 
Anleitung für \emph{eduroam} durchlesen oder dort die
vorkonfigurierte Clientsoftware AnyConnect herunterladen, welche dich
nach Anmelden mit deiner Campuskennung in ein VPN (Virtual Private
Network) des LRZ einbucht. Aus Netzwerksicht verhält sich dein Rechner
dann wie alle anderen Rechner im Münchener Wissenschaftsnetz. So
kannst du nicht nur normal surfen, sondern auch von außen auf das
Münchner Wissenschaftsnetz zugreifen und zum Beispiel bestimmte
Artikel aus der Bibliothek lesen.

Die Clientsoftware ist übrigens außerhalb der Uni praktisch, um deine
HTTP-Verbindungen zu verschlüsseln, etwa wenn du dich in einem
ungeschützen WLAN befindest.

\subsection*{Microsoft DreamSpark \subjectList{\subjectI{}\subjectMI{}\subjectP{}}}
Studika der Physik und Informatik (auch im Nebenfach) bekommen über
Microsoft DreamSpark (früher MSDNAA) viele Microsoftproduktlizenzen
gratis, darunter Windows, Visual Studio und viele
Microsoft Office-Komponenten, jedoch \textbf{nicht} Word, Excel und PowerPoint.

\begin{urlList}
	\httpItem[Informatik]{tools.rz.ifi.lmu.de/cipconf/index.rb?op=msdnaa}
	\httpsItem[Physik]{msdnaa.physik.uni-muenchen.de/}
\end{urlList}

\subsection*{Proxyeinrichtung}
Um E-Books, Paper, und wissenschaftliche
Journale der Uni-Bib herunterladen zu können, benötigst du einen Proxyzugang.
Dazu musst du entweder im MWN (Münchner Wissenschaftsnetz, z.B. im CIP) sitzen
oder EasyProxy verwenden. Wer sich schon etwas auskennt: ssh in einen CIP mit X11-forwarding
eines Browsers geht auch.
\begin{urlList}
	\httpItem[EasyProxy Anleitung]{www.ub.uni-muenchen.de/elektronische-medien/hilfe-anleitungen/easyproxy/}{proxy}
\end{urlList}
