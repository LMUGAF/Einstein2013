
\section{Vorlesungszeit}

\subsection{Wie bastelt man einen Stundenplan?}

Das Vorlesungsverzeichnis findest du unter: \url{www.lsf.lmu.de}

Auf den Fakultätsseiten gibt es zum Teil noch eine kommentierte
Version. In der Aktualität wechseln sich die beiden Seiten ab, meist
ist die kommentierte Version jedoch vollständiger.

Die Erstellung über das lsf ist sehr kompliziert und weniger
empfehlenswert.  Wichtig ist, dsas du explizit auf ``Plan Speichern''
klickst!  Papier vergisst nichts.

\begin{itemize}
	\item Sortiere nach verbindlichen und empfohlenen Veranstaltungen.
	\item Plane erst obligatorische Veranstaltungen (Studienplan).
	\item Beachte Lehrveranstaltungszyklen (Was baut aufeinander auf?)
	\item Beachte, ob eine Lehrveranstaltung nicht in jedem
          Semester angeboten (z.b. nur jährlich) wird und ob
          Vorlesungen und Seminare oder Übungen im Zusammenhang
          stehen.
	\item Plane Lehrveranstaltungen in einem Umfang von höchstens 20 Semesterwochenstunden, denn Übungen, Tutorien, Selbststudienzeiten sowie Vor- und Nachbereitungen sind in jedem Fall notwendig.
	\item Beachte Wege und Fahrzeiten zwischen den Vorlesungen.
	\item Überprüfe den Stundenplan nach der ersten Vorlesungswoche in Bezug auf Mach- und Brauchbarkeit deines Plans.
	\item Erstelle darüber hinaus einen Semesterplan, in dem alle Termine, Fristen, Aktivitäten vermerkt sind, wie Rückmeldefristen, Klausuren, Referate oder Vorbereitungszeiten für Prüfungen.
	\item Schaue über den Tellerrand hinaus und tief in den Teller hinein. Die LMU bietet eine Vielzahl von Studiengängen an. Suche dir ruhig auch einmal etwas heraus, was dich zwar interessiert, du aber nicht in dein Studium einbringen kannst (von Arabistik bis Zoologie\ldots).
	\item Berücksichtige auch zusätzliche Veranstaltungen, wie beispielsweise Sprachen lernen, Computerkurse, Sport o.ä.
\end{itemize}


\subsection{Zusatz-Angebote}
\begin{itemize}
	\item Fremdsprachen: \url{www.sprachenzentrum.lmu.de}
	\item Ringvorlesung: \url{www.lmu.de/ringvorlesung}
	\item Studium Generale: \newline \url{www.lmu.de/studium/studienangebot/lehrangebote/studium_generale}
	\item LMU PLUS Seminare: \url{www.frauenbeauftragte.lmu.de/plus/plus_veranstaltungen}
	\item Soft Skills, Bewerbungstraining: \url{s-a.uni-muenchen.de} und \url{www.jobline.lmu.de}
	\item Soft Skills an der TUM: \newline \url{www.cvl-a.de/index.php?option=com_content&view=article&id=24}
\end{itemize}


\clearpage


\subsection{Zentraler Hochschulsport (ZHS)}
Für den körperlichen Ausgleich zum Studium kann man in kostspielige Fitnesscenter gehen oder aber eine der vielen interessanten Sportarten, wie z.B. Fechten, Segeln oder Bergsteigen ausprobieren, die vom ZHS zu einem relativ günstigen Preis (ab 7,50~€ pro Semester) angeboten werden. Der Großteil des Angebots findet auf dem Olympiagelände statt und ist -- abgesehen vom Fahrrad -- am besten mit der U3 (Haltestelle Olympiazentrum) und einem kurzen Fußmarsch durchs Olympische Dorf zu erreichen.

Das komplette Sportangebot könnt ihr der Homepage (\url{zhs-muenchen.de}) und dem Hochschulsportheft entnehmen, das zu Semesterbeginn unter anderem im Gumbel ausliegt.

Meistens ist eine Onlineanmeldung verpflichtend, damit du an den
Kursen teilnehmen darfst. Bringe deine Anmeldebestätigung ausgedruckt
mit. Für die Teilnahme brauchst du einen ZHS-Ausweis der
entsprechenden Kategorie mit gültigen Sportmarken, welche online
gebucht werden müssen. Danach musst du dir mit ausgedruckter
Buchungsbestätigung, Studentenausweis, Lichtbildausweis und Passfoto
in der ZHS einen Ausweis erstellen lassen und die entsprechenden
Marken besorgen. In der ersten Woche des Semesters ist das auch in der
Innenstadt (Schellingstr. 3) möglich.



\subsection{Essen}

Die verschiedenen Mensen des Studentenwerks mit Speiseplänen findet ihr unter\\ \url{studentenwerk-muenchen.de/mensa}. Zum Bezahlen braucht man eine Mensakarte, die man dort erwerben und aufladen kann.

In manchen Universitätsgebäuden ist darüber hinaus eine Cafeteria zu finden mit ähnlich preiswerten Essensangebot (Hauptgebäude (HGB) Nordhof, Schellingstr. 1. Stock, Oettingenstr. Keller, Giselastr. Mensagebäude).

Auch wenn du dir selbst ein Bild machen solltest, hier vorab ein Testbericht der Süddeutschen zu den verschiedenen Mensen:
sueddeutsche.de/muenchen/uni-mensen-im-test-\newline voll-auf-die-geschmacksnerven-1.32268

Wenn dir das Essen in den Mensen auf Dauer zu langweilig wird und du trotzdem nicht viel Geld ausgeben willst, hier ein paar Geheimtipps:

\begin{itemize}
	\item \textbf{Finanz- bzw. Landwirtschaftsministerium} (Odeonsplatz 4 bzw. Ludwigstr. 2): Ausschließlich für Mitarbeitern und Studenten. Darum musst du auch einen gültigen (Münchener) Studentenausweis und manchmal zusätzlich deinen Personalausweis vorzeigen. Es gibt täglich wechselnde Gerichte zu Preisen von 3,90~€ bis 6,00~€, jeden Mittwoch ist der allseits beliebte “Schnitzeltag” (4,10~€ mit Salat und Beilage).

	\item \textbf{HFF-Mensa (Hochschule für Film und Fernsehen)}
          (Bernd-Eichinger-Platz 1, gegenüber der TUM-Mensa): Etwas
          teurer als unsere Mensa, dafür aber besser.
\end{itemize}
