\chapter{Vorlesungszeit}

\section{Der Stundenplan}

Dein persönlicher Stundenplan gehört zu den wichtigsten Dingen, um die du dich jedes Semster neu kümmern musst.
Für das erste Semester geben wir dir einen Musterstundenplan an die Hand, allerdings solltest du beachten, dass darin noch keine Übungen und Nebenfachvorlesungen enthalten sind. Also unterschätze nicht, wenn dein Stundenplan am Anfang sehr leer aussieht, denn die Löcher füllen sich schneller auf als du denkst.

Um einen eigenen Stundenplan zu erstellen, gibt es einige Dinge, die du wissen
musst. Das Vorlesungsverzeichnis findest du unter \ref{lsf}, dort gibt es auch
ein Tool ``Stundenplan'', in dem du deine Vorlesungen speichern kannst.
Aufgrund der etwas merkwürdigen Bedienung des Ganzen solltest du nach jeder
Änderung an deinem Plan auf ``Plan Speichern'' klicken, sonst fehlt
möglicherweise etwas. Falls du ganz auf Nummer sicher gehen willst, schreibe den
Plan lieber von Hand, denn Papier vergisst nichts, dazu haben wir dir auf Seite
\pageref{studenplan} gleich eine Vorlage hier ins Heft gedruckt. Sobald du
diese Hürde genommen hast, stellt sich natürlich das Problem, womit man den
Plan füllt. Hier ein paar Tipps:

\begin{itemize}
	\item Unterscheide Pflicht- und Wahlveranstaltungen (aus dem Studienplan zu entnehmen)
	\item Plane zuerst Pflichtveranstaltungen (Bedenke den Zyklus, z.B. jährlich)
	\item Welche Veranstaltungen bauen aufeinander auf?
	\item Welche Veranstaltungen gehören zusammen? (Vorlesungen, Seminare, Tutorien oder Übungen)
	\item Plane maximal 20 Wochenstunden für Vorlesungen und Seminare ein (Lasse Platz für Übungen, Tutorien, Vor- und Nachbereitungen)
	\item Achte auf die Veranstaltungsorte (komme ich in 15 Minuten von A nach B?)
	\item Besuche interessante Veranstaltungen aus anderen Fachbereichen (von Arabistik bis Zoologie)
	\item Nach der ersten oder zweiten Vorlesungswoche: Prüfe, ob es nicht zu viel ist
\end{itemize}

Auf manchen Fakultätsseiten gibt es ein kommentiertes Vorlesungsverzeichnis, das dir etwas bei der Auswahl der Veranstaltungen helfen kann und meist aktueller als das zentrale Vorlesungsverzeichnis ist.
Erstelle darüber hinaus einen Semesterplan, in dem alle wichtigen Termine vermerkt sind wie Rückmeldefristen, Klausuren, Referate oder Vorbereitungszeiten für Prüfungen. %TODO Link zu unserem Kalender einfügen

\begin{urlList}
	\urlItem{http://www.lsf.lmu.de}[lsf]
\end{urlList}

	
%Das hier als Link-Liste anlegen? sont weglassen
%\section{Zusatz-Angebote}
%\begin{itemize}
%	\item Fremdsprachen: \url{www.sprachenzentrum.lmu.de}
%	\item Ringvorlesung: \url{www.lmu.de/ringvorlesung}
%	\item Studium Generale: \newline \url{www.lmu.de/studium/studienangebot/lehrangebote/studium_generale}
%	\item LMU PLUS Seminare: \url{www.frauenbeauftragte.lmu.de/plus/plus_veranstaltungen}
%	\item Soft Skills, Bewerbungstraining: \url{s-a.uni-muenchen.de} und \url{www.jobline.lmu.de}
%	\item Soft Skills an der TUM: \newline \url{www.cvl-a.de/index.php?option=com_content&view=article&id=24}
%\end{itemize}




\section{Zentraler Hochschulsport (ZHS)}

Für den körperlichen Ausgleich zum Studium kannst du in kostspielige
Fitnesscenter gehen oder aber eine der vielen interessanten Sportarten
ausprobieren, die vom ZHS zu einem relativ günstigen Preis (ab 7,50~€ pro
Semester) angeboten werden. Der Großteil des Angebots findet auf dem
Olympiagelände statt und ist (abgesehen vom Fahrrad) am besten mit der U3
(Haltestelle Olympiazentrum) und einem kurzen Fußmarsch durchs Olympische Dorf
zu erreichen. Für die Teilnahme brauchst du einen ZHS-Ausweis der
entsprechenden Kategorie mit gültigen Sportmarken, welche online unter
\ref{marken} gebucht werden müssen. Danach musst du dir mit ausgedruckter
Buchungsbestätigung, Studentenausweis, Lichtbildausweis und Passfoto einen
Ausweis erstellen lassen und die entsprechenden Marken besorgen. In der ersten
Woche des Semesters ist das in der Innenstadt (Schellingstr. 3, Leopoldstr. 13)
möglich, die restliche Zeit im ZHS im Olympiazentrum. 

Der ZHS bietet ein breites Spektrum an Sportarten mit sehr unterschiedlichen
Anforderungen (Anfänger, Fortgeschrittene, freies Training\ldots). Das
komplette Sportangebot könnt ihr \ref{zhs} und dem Hochschulsportheft
entnehmen, das zu Semesterbeginn unter anderem im Gumbel ausliegt. Für viele
Kurse ist eine Onlineanmeldung nur formal verpflichtend, um daran teilnehmen zu
dürfen, beachte aber, dass es Sportarten gibt, die sehr beliebt und deshalb
schnell belegt sind (z.B. Segeln oder Bergsteigen). Bringe zu solchen
Veranstaltungen sicherheitshalber deine Anmeldebestätigung mit.

\begin{urlList}
	\urlItem{http://zhs-muenchen.de}[zhs]
	\urlItem{http://sportan3.zhs.ze.tum.de/angebote/aktueller_zeitraum_0/index_marken.html}[marken]
\end{urlList}

\section{Musik}
Falls du auch mal etwas anderes auf die Ohren brauchst als eine
Mütze Schlaf, finden sich an der Uni in der Regel immer Leute, die gerne Musik
machen und sei die Musikrichtung noch so absurd. Einen Überblick über die
etablierteren Gruppen findest du unter \ref{musik}, ansonsten helfen Google und
Aushänge weiter. Trau dich einfach, verschiedene Sachen auszuprobieren, denn
auf Anhieb das Richtige zu finden ist eher schwer. Sobald man aber Leute kennt,
wird es wesentlich einfacher.

\begin{urlList}
	\urlItem{http://www.uni-muenchen.de/studium/stud_leben/kulturelles-leben/index.html}[musik]
\end{urlList}

\section{Kino}
Auch für filmische Unterhaltung ist gesorgt, sowohl von Seiten der LMU als auch
der TUM. Während der tu film Blockbuster zeigt, liegt der Fokus des u.kinos
eher auf Perlen abseits des Mainstreams.

\begin{urlList}
	\urlItem{http://u-kino.de}[ukino]
	\urlItem{http://tu-film.de}[tufilm]
\end{urlList}

\section{Essen}
Die verschiedenen Mensen des Studentenwerks mit Speiseplan findest du unter \ref{mensa}. 
Zum Bezahlen brauchst du eine Mensakarte, die du dort oder bei uns während der O"~Phase für 12~€ erwerben kannst (davon 7~€ Pfand). In manchen Universitätsgebäuden gibt es eine Cafeteria mit ähnlich preiswerten Essensangeboten, aber etwas längeren Öffnungszeiten, die man außerhalb der Mittagszeit auch als Aufenthaltsraum nutzen kann (Hauptgebäude Nordhof, Schellingstr. 3 (1. Stock), Oettingenstr. (Keller), Mensagebäude Leopoldstr.).

Wenn dir das Essen in den Mensen auf Dauer zu langweilig wird und du trotzdem nicht viel Geld ausgeben willst, hier ein paar Geheimtipps:

\begin{itemize}
	\item \textbf{Finanz- bzw. Landwirtschaftsministerium} (Odeonsplatz 4 bzw. Ludwigstr. 2): mit Studentenausweis und evtl. Personalausweis täglich wechselnde Gerichte zu Preisen von 3,90~€ bis 6,00~€, jeden Mittwoch Schnitzeltag (4,10~€ mit Salat und Beilage).
	
	\item \textbf{HFF-Mensa (Hochschule für Film und Fernsehen)}
          (Bernd-Eichinger-Platz 1, gegenüber der TUM-Mensa): etwas
          teurer als unsere Mensa, dafür aber besser.
\end{itemize}

\begin{urlList}
	\urlItem{http://studentenwerk-muenchen.de/mensa}[mensa]
\end{urlList}
