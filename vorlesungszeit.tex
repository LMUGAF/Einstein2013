\chapter{Vorlesungszeit}

\section{Veranstaltungen}

Es gibt an der Uni verschiedene Arten von Veranstaltungen, von Vorlesungen und Tutorien über Proseminare und Kolloquia bis hin zu  Sprach- und Lektürekursen. Die wichtigsten sind:

\begin{itemize}
	\item Vorlesungen: Hier präsentiert ein Professorikon den (oft prüfungsrelevanten) Stoff. Vorlesungen benötigen einiges an Vor- und Nachbereitung, also wundere dich nicht, wenn du nicht alles auf Anhieb verstehst.
	\item Übungen, Tutorien o.Ä.: In kleineren Gruppen wird das in der Vorlesung Gelernte wiederholt und angewandt. Meist werden hier die Übungsblätter besprochen.
	\item Seminare: Im Gegensatz zu den Vorlesungen tragen hier die Studika vieles selbst bei, zum Beispiel durch Präsentationen und Diskussionen.
\end{itemize}

Außerdem musst du zwischen Pflicht- und Wahlpflichtveranstaltungen unterscheiden.

\section{Der Stundenplan}

Dein persönlicher Stundenplan gehört zu den wichtigsten Aufgaben, um die du dich jedes Semster neu kümmern musst. Für das erste Semester geben wir dir einen Musterstundenplan an die Hand, allerdings solltest du beachten, dass darin noch keine Übungen und Nebenfachvorlesungen enthalten sind. Also unterschätze nicht, wenn dein Stundenplan am Anfang sehr leer aussieht, denn die Löcher füllen sich schneller als du denkst.

Das Erstellen des Stundenplans ist etwas umständlich. Zuerst musst du die Veranstaltung, die du hinzufügen möchtest, im LSF finden (``Suche nach Veranstaltungen'') und vormerken (dazu gibt es ein kleines Kästchen, direkt unter den Terminen). Sie wird nun in deinem Stundenplan angezeigt. Diesen kannst du abspeichern, auf dein Handy laden oder dir als PDF ausdrucken. Da das Tool manchmal nicht das tut, was man will, haben wir dir auf Seite \pageref{studenplan} eine Vorlage ins Heft gedruckt, in die du deinen Stundenplan per Hand eintragen kannst.

Hast du einmal herausgefunden, \textit{wie} du deinen Stundenplan erstellst, stellt sich natürlich die Frage womit du ihn füllst. Auf der Homepage findest du unter der Beschreibung deines Studiengangs eine Übersicht über die Veranstaltungen, die du im Laufe deines Studiums belegen musst. Plane am besten zuerst alle Pflichtveranstaltungen (dabei solltest du beachten, dass einige davon nur alle zwei Semester angeboten werden) und kümmere dich dann erst um die anderen Veranstaltungen. Du solltest nicht mehr als 20 Wochenstunden für Vorlesungen und Seminare aufwenden und so genügend Platz für Übungen, Tutorien und die Vor- und Nachbereitung lassen.

Da es an der LMU sehr viele verschiedene Studiengänge gibt, ist es nur empfehlenswert, interessante Veranstaltungen aus anderen Fachbereichen zu besuchen. Außerdem gibt es Sprachkurse und viele andere Angebote. Diese findest du unter ``Zusatzqualifikationen für Studierende''. Nach der ersten oder zweiten Vorlesungswoche solltest du schließlich prüfen, ob es dir nicht zu viel ist.

Hast du deinen Stundenplan fertig erstellt, solltest du dir noch einen Semesterplan anlegen, in dem alle wichtigen Termine wie Rückmeldefristen, Klausuren, Referate oder Vorbereitungszeiten für Prüfungen vermerkt sind. Diese kannst du auch in unseren Kalender in der Mitte des Einsteins eintragen.

%Benennen der URLS??

\begin{urlList}
	\urlItem{http://www.lsf.lmu.de}[lsf]
	\urlItem{http://www.hilfe.lsf.uni-muenchen.de/lsf_hilfe/funktionen/stdplan/index.html}[stundenplan]
	\urlItem{http://www.uni-muenchen.de/studium/studienangebot/zusatzquali/index.html}[zusatzqualifikationen]
\end{urlList}

	
%Das hier als Link-Liste anlegen? sont weglassen
%\section{Zusatz-Angebote}
%\begin{itemize}
%	\item Fremdsprachen: \url{www.sprachenzentrum.lmu.de}
%	\item Ringvorlesung: \url{www.lmu.de/ringvorlesung}
%	\item Studium Generale: \newline \url{www.lmu.de/studium/studienangebot/lehrangebote/studium_generale}
%	\item LMU PLUS Seminare: \url{www.frauenbeauftragte.lmu.de/plus/plus_veranstaltungen}
%	\item Soft Skills, Bewerbungstraining: \url{s-a.uni-muenchen.de} und \url{www.jobline.lmu.de}
%	\item Soft Skills an der TUM: \newline \url{www.cvl-a.de/index.php?option=com_content&view=article&id=24}
%\end{itemize}


\section{Zentraler Hochschulsport (ZHS)}

Für den körperlichen Ausgleich zum Studium kannst du in kostspielige
Fitnesscenter gehen oder aber eine der vielen interessanten Sportarten
ausprobieren, die vom ZHS zu einem relativ günstigen Preis (ab 7,50~€ pro
Semester) angeboten werden. Der Großteil des Angebots findet auf dem
Olympiagelände statt und ist (abgesehen vom Fahrrad) am besten mit der U3
(Haltestelle Olympiazentrum) und einem kurzen Fußmarsch durchs Olympische Dorf
zu erreichen. Für die Teilnahme brauchst du einen ZHS-Ausweis der
entsprechenden Kategorie mit gültigen Sportmarken, welche online unter
\ref{marken} gebucht werden müssen. Danach musst du dir mit ausgedruckter
Buchungsbestätigung, Studentenausweis, Lichtbildausweis und Passfoto einen
Ausweis erstellen lassen und die entsprechenden Marken besorgen. In der ersten
Woche des Semesters ist das in der Innenstadt (Schellingstr. 3, Leopoldstr. 13)
möglich, die restliche Zeit im ZHS im Olympiazentrum. 

Der ZHS bietet ein breites Spektrum an Sportarten mit sehr unterschiedlichen
Anforderungen (Anfänger, Fortgeschrittene, freies Training\ldots). Das
komplette Sportangebot könnt ihr \ref{zhs} und dem Hochschulsportheft
entnehmen, das zu Semesterbeginn unter anderem im Gumbel ausliegt. Für viele
Kurse ist eine Onlineanmeldung nur formal verpflichtend, um daran teilnehmen zu
dürfen, beachte aber, dass es Sportarten gibt, die sehr beliebt und deshalb
schnell belegt sind (z.B. Segeln oder Bergsteigen). Bringe zu solchen
Veranstaltungen sicherheitshalber deine Anmeldebestätigung mit.

\begin{urlList}
	\urlItem{http://zhs-muenchen.de}[zhs]
	\urlItem{http://sportan3.zhs.ze.tum.de/angebote/aktueller_zeitraum_0/index_marken.html}[marken]
\end{urlList}

\section{Musik}
Falls du auch mal etwas anderes auf die Ohren brauchst als eine
Mütze Schlaf, finden sich an der Uni in der Regel immer Leute, die gerne Musik
machen und sei die Musikrichtung noch so absurd. Einen Überblick über die
etablierteren Gruppen findest du unter \ref{musik}, ansonsten helfen Google und
Aushänge weiter. Trau dich einfach, verschiedene Sachen auszuprobieren, denn
auf Anhieb das Richtige zu finden ist eher schwer. Sobald man aber Leute kennt,
wird es wesentlich einfacher.

\begin{urlList}
	\urlItem{http://www.uni-muenchen.de/studium/stud_leben/kulturelles-leben/index.html}[musik]
\end{urlList}

\section{Kino}
Auch für filmische Unterhaltung ist gesorgt, sowohl von Seiten der LMU als auch
der TUM. Während der tu film Blockbuster zeigt, liegt der Fokus des u.kinos
eher auf Perlen abseits des Mainstreams.

\begin{urlList}
	\urlItem{http://u-kino.de}[ukino]
	\urlItem{http://tu-film.de}[tufilm]
\end{urlList}

\section{Essen}
Die verschiedenen Mensen des Studentenwerks mit Speiseplan findest du unter \ref{mensa}. In manchen Universitätsgebäuden gibt es eine Cafeteria mit ähnlich preiswerten Essensangeboten, aber etwas längeren Öffnungszeiten, die man außerhalb der Mittagszeit auch als Aufenthaltsraum nutzen kann (Hauptgebäude Nordhof, Schellingstr. 3 (1. Stock), Oettingenstr. (Keller), Mensagebäude Leopoldstr.).

Zum Bezahlen brauchst du eine Mensakarte, die du in einer der Mensen oder bei uns während der O"~Phase für 12~€ erwerben kannst (davon 7~€ Pfand). Am besten schreibst du dir gleich die Nummer deiner Mensa-Karte auf, dann bekommst du beim Verlust der Karte das Geld, das gerade drauf war, ausgezahlt.

Wenn dir das Essen in den Mensen auf Dauer zu langweilig wird und du trotzdem nicht viel Geld ausgeben willst, hier ein paar Geheimtipps:

\begin{itemize}
	\item \textbf{Finanz- bzw. Landwirtschaftsministerium} (Odeonsplatz 4 bzw. Ludwigstr. 2): mit Studentenausweis und evtl. Personalausweis täglich wechselnde Gerichte zu Preisen von 3,90~€ bis 6,00~€, jeden Mittwoch Schnitzeltag (4,10~€ mit Salat und Beilage).
	
	\item \textbf{HFF-Mensa (Hochschule für Film und Fernsehen)}
          (Bernd-Eichinger-Platz 1, gegenüber der TUM-Mensa): etwas
          teurer als unsere Mensa, dafür aber besser.
\end{itemize}

\begin{urlList}
	\urlItem{http://studentenwerk-muenchen.de/mensa}[mensa]
\end{urlList}
