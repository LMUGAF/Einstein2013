\chapter{Vorlesungsfreie Zeit}

\ldots heißt deshalb nicht Ferien, sondern vorlesungsfreie Zeit, weil man hier
endlich die Zeit hat in Ruhe zu lernen, Klausuren zu schreiben und
Blockseminare sowie Praktika zu besuchen. Die Uni kalkuliert die
Arbeitsbelastung so, dass man übers Jahr gerechnet etwa 6 Wochen Ferien hat, wie
ein normaler Arbeitnehmer auch.

\section{Klausuren und Protokolle}
Wir von der GAF sammeln Altklausuren und mündliche Prüfungsprotokolle. Das
meiste davon findest du in unserer Online-Sammlung \ref{klausuren}.
Benutzername und Passwort kannst du bei uns in der GAF erfragen oder dir per
E-Mail \ref{daten} zuschicken lassen.  Es existieren auch noch einzelne
ungescannte Protokolle von mündlichen Prüfungen, die du zum Kopieren ausleihen
kannst, vor allem von Physikern, die relativ selten gefragt wurden. Im
Zweifelsfall schaue zuerst online nach, falls du dort nichts findest, kannst
du gerne in unseren Ordnern suchen.

Damit auch künftige Generationen davon profitieren, schicke bitte alles,
was du in die Hände bekommst, (sofern noch nicht vorhanden) an uns.
Wenn du in einer Klausur sitzt, in der die Offiziellen mit Strafen
drohen, wenn jemand die Klausuren mitnimmt/abschreibt, erstellt
direkt im Anschluss ein Gedächtnisprotokoll.

Die Nächsten werden es dir danken!

\begin{urlList}
	\urlItem{https://gaf.fs.lmu.de/klausuren}[klausuren]
	\urlItem{https://gaf.fs.lmu.de/zugangsdaten/zugangsdaten.html}[daten]
\end{urlList}

