
\section{Ankommen in M�nchen}

\subsection{Ummeldung / Zweitwohnsitz}
\begin{itemize}
	\item Ort: \url{muenchen.de/dienstleistungsfinder/muenchen/1063475/}
	\item Zweitwohnungsteuer: 9 \% der j�hrlichen Nettokaltmiete
	\item Befreiung Zweitwohnsitzsteuer: Nachweis �ber positive Eink�nfte unter 25.000~� (z.B. BAf�G-Bescheid) bis 31.1. \newline (\url{muenchen.de/Rathaus/ska/fachinfos/zweitwohnst/einkommensabhaengig.html})
\end{itemize}

\subsection{M�lltrennung}
\begin{itemize}
	\item Trennung nach: Restm�ll, Papier, (Bio,) Plastik, Dosen, Altglas und den �blichen Sachen wie Sperrm�ll, Elektroschrott u.�.
	\item F�r Plastik, Dosen und Altglas gibt es Container, die �ber die Stadt verteilt sind
	\item \url{awm-muenchen.de}
\end{itemize}

%\subsection{Wohnung}
%\begin{itemize}
%	\item Die beliebtesten Portale f�r WGs: \url{wg-gesucht.de} und \url{studenten-wg.de} (Wenn ihr euch bewerbt, schreibt mehr als euren Namen und Telefonnummer.  Stellt euch pers�nlich vor, bringt Bier mit ;-)).
%	\item Wohnheim: \url{studentenwerk-muenchen.de/wohnen}
%	\item andere Angebote: \url{studentenwerk-muenchen.de/wohnen/weitere-wohnangebote}
%	\item Notunterk�nfte: \url{www.caritastoelz.de/Page010179.htm}, \url{www.wohnhilfe-muenchen.de/jugendhilfe/die-jugendpension-jup.html}
%\end{itemize}

\subsection{GEZ}
\begin{itemize}
	\item Nach der Ummeldung wirst du wahrscheinlich bald Post von der GEZ bekommen.
	\item Befreiung m�glich, wenn du BAf�G bekommst (Antrag stellen)
	\item Geb�hren Radio, Computer: 5,76~�/Monat \\
Geb�hren Radio, Computer, Fernseher: 17,98~�/Monat
        \item \url{gez.de}
        \item http://www.lawblog.de/index.php/archives/2011/04/15/generelles-hausverbot-fr\newline -gez-mitarbeiter-mglich/
\end{itemize}

\subsection{Rundfunkbeitrag}
\begin{itemize}
    \item ab 1.1.2013
    \item 17,98~�/Monat pro Haushalt unabh�ngig von den Ger�ten
    \item Befreiungsantrag ab November m�glich
    \item \url{rundfunkbeitrag.de}
\end{itemize}

\subsection{Wohnen}
Teuer, schwer zu bekommen und hart umk�mpft. Die Mietpreise liegen
auch f�r Studenten ca. 50--100~� �ber dem �blichen mittleren Preis in
Restdeutschland. Das Studentenwerk bietet auf seiner Homepage eine
gute �bersicht �ber

\begin{itemize}
\item Studentenwerkswohnheime \newline \url{studentenwerk-muenchen.de/wohnen/wohnanlagen-des-studentenwerks-muenchen/}

  g�nstig aber schwer zu bekommen (erkundigt euch in den Verwaltungstellen direkt)
\item private Wohnheime, oft in eigener oder karitativer Tr�gerschaft
  (Bewerbungen sind n�tig, z. T. gibt es Bedingungen, der Versuch lohnt sich)
\item
  Privatzimmer \newline \url{http://www.studentenwerk-muenchen.de/wohnen/vermittlung-von-privatzimmern/}
  
werden vom Studentenwerk und der Mitwohnzentrale vermittelt.
\item Wohnen gegen Hilfe f�r �ltere Leute, die der helfenden Hand daf�r
  Wohnraum stellen.
\item andere Angebote: \url{studentenwerk-muenchen.de/wohnen/weitere-wohnangebote}
\item {\bf Notunterk�nfte}, falls alles schiefgeht:

 \url{www.caritastoelz.de/Page010179.htm},

  \url{www.wohnhilfe-muenchen.de/jugendhilfe/die-jugendpension-jup.html}
\end{itemize}

\paragraph{WGs gibt es reichlich, sucht hier:}
\url{wg-gesucht.de} und \url{studenten-wg.de}\newline
Eine freundliche E-Mail mit einer Vorstellung Eurer selbst und warum ihr
in diese WG passt ist wichtig.

\paragraph{Selbst mieten} ist teuer, aufw�ndig und oft sind Provisionen
f�llig. Die Mieten sind in den letzten Jahren nochmal kr�ftig gewachsen.
Suchen
lohnt sich in den g�ngigen Online Portalen und auf der Immobilienseite der
S�ddeutschen Zeitung, auch online. Meistens werden B�rgschaften oder andere
Sicherheiten verlangt.  Wer vorbereitet zur Besichtigung kommt, ist im Vorteil.
