
\section{Hilfe und Beratung}

\subsection{Erste Hilfe: GAF}

Wir kennen nicht immer die Lösung, wissen dafür aber meistens, wer sie
kennt. Wir haben gute Kontakte zu vielen Institutionen und Personen an
dieser Uni. Wenn du uns einfach mal besuchen willst, bist du herzlich
willkommen. (Kontakt siehe \ref{gafKontakt}, S. \pageref{gafKontakt})

\subsection{Probleme mit Lehrveranstaltungen oder Lehrpersonal}

Die offizielle Ansprechstelle hierbei ist der Studiendekan deiner
Fakultät.  Er ist für die Qualität der Lehre verantwortlich. In jedem
Fall ist der sinnvollste Weg zu einer Lösung erst einmal das direkte
Gespräch mit dem Dozenten. Erst wenn ihr das Gefühl habt, ein Problem
lässt sich nicht anders lösen, bittet euren Studiendekan um
Hilfe. Oder fragt uns von der GAF.

\paragraph{Für die Fakultät 16 (Mathe, Info, Statistik)}\hfill\\
Prof. Werner Bley\\
Prof. Hans Jürgen Ohlbach\\
Prof. Thomas Augustin

\paragraph{Für die Fakultät 17 (Physik und Meteorologie)}\hfill\\
Prof. Erwin Frey

\subsection{Webforen}

In den Foren \url{die-informatiker.net}, \url{die-physiker.org}
und \url{die-mathematiker.net} kann man sich mit seinen Kommilitonen
(und teils auch Lehrpersonal) austauschen.


\subsection{Ansprechpartner}

Alle nachfolgenden Personen sind sehr umgängliche Menschen, mit denen
man bestens reden kann. Wie die meisten Professoren beißen sie nicht,
wenn man was zu beanstanden hat.

\paragraph{Mathematik (B.Sc., LA Gymnasium)}\hfill\\
Dr. Heribert Zenk (Heribert.Zenk@mathematik.uni-muenchen.de)\\
Theresienstr. 39, B333, Telefon: 089 / 2180\textendash 4460\\
Sprechstunde: Mo, 15:00\emd{}16:00~Uhr\\

Dr. Hartmut Weiß (hartmut.weiss@mathematik.uni-muenchen.de)\\
Theresienstr. 39, B317, Telefon: 089 / 2180\emd{}4680\\
Sprechstunde: Do, 15:00\emd{}16:00~Uhr

\paragraph{Wirtschaftsmathematik (B.Sc.)}\hfill\\
Prof. Dr. Gregor Svindland (studienberatung.wirtschaftsmathematik@math.lmu.de)\\
Theresienstr. 39, B231, Telefon: 809 / 2180\emd{}4628\\
Sprechstunde: nach Vereinbarung

\paragraph{Mathematik (LA Grund-, Haupt-, und Realschule)}\hfill\\
Dr. Erwin Schörner (schoerner@lmu.de)\\
Theresienstr. 39, B237, Telefon: 089 / 2180\emd{}4498\\
Sprechstunde: nach Vereinbarung

\paragraph{Mathematik (Fachdidaktik und Didaktik)}\hfill\\
Prof. Dr. Hedwig Gasteiger (gasteiger@math.lmu.de)\\
Theresienstr. 39, B215, Telefon: 089 / 2180\emd{}4631\\
Sprechstunde: nach Vereinbarung

\paragraph{Informatik (B.Sc.)}\hfill\\
Dr. Reinhold Letz (reinhold.letz@lmu.de)\\
Oettingenstr. 67, E001, Telefon: 089 / 2180\emd{}9693\\
Sprechstunde: Mo \& Do 13:00\emd{}14:00~Uhr und nach Vereinbarung

\paragraph{Informatik (LA)}\hfill\\
Prof. Martin Hofmann, PhD (lehramt@ifi.lmu.de)\\
Oettingenstr. 67, L107, Telefon: 089 / 2180\emd{}9341

\paragraph{Medieninformatik (B.Sc.)}\hfill\\
Max Maurer / Simon Stusak (studentenbetreuer@medien.ifi.lmu.de)\\
Amalienstr. 17, 505, Telefon: 089 / 2180\emd{}4654

\paragraph{Physik (B.Sc.)}\hfill\\
Dr. Jana Traupel (jana.traupel@physik.uni-muenchen.de)\\
Schellingstr. 4, H417, Telefon: 089 / 2180\emd{}5033

\paragraph{Physik plus Meteorologie (B.Sc.)}\hfill\\
Dipl. Met. Heinz Lösslein (loesslein@lmu.de)\\
Theresienstr. 37, A208, Telefon: 089 / 2180\emd{}4217

\paragraph{Physik (LA)}\hfill\\
Prof. Dr. Raimund Girwidz (girwidz@physik.uni-muenchen.de)\\
Theresienstr. 37, A012, Telefon: 089 / 2180\emd{}2020



\subsection{Prüfungsamt}
Die Prüfungsämter sind für alle Prüfungsangelegenheiten zuständig,
d.h. deine Noten, deine Praktika, deine Notenübersichten und
Abschlusszeugnisse etc. Sie sind bei der Fakultät zu finden, zu der
dein Studienfach gehört. Eine Zuordnung der Prüfungsämter zu den
einzelnen Studiengängen/-fächern findest du auf der Übersichtsseite
Studiengänge A--Z am unteren Ende der jeweiligen
Studiengangsinformationen.

\url{www.lmu.de/pruefungsaemter}

\clearpage

\subsection{Studentenkanzlei}

Die Studentenkanzlei muss wegen gewissen formalen Belangen
gelegentlich besucht werden. Der Besuch dieses kafkaesken Molochs ist
oft mit großen Wartezeiten und Unbill verbunden.  Es hilft, hartnäckig
zu bleiben und notfalls mehrfach zu kommen, bis man den richtigen
Sachbearbeiter trifft. Nicht umgehen lässt sich ein Besuch bei:

\begin{itemize}
\item Beantragen von Beurlaubungen (Krankheit, Ausland, Kinder\ldots)
\item Fragen zur Studienplatzvergabe/Immatrikulation (Anerkennung von Hochschulzugangsberechtigungen, nachträgliches Einschreiben, Verlust der Immatrikulationsbescheinigung)
\item Studienfachwechsel, zusätzliche Einschreibung für ein Doppelstudium
\item Bescheinigungen für die Krankenkasse, Rente, Quittungen für Studienbeiträge
\end{itemize}

HGB, E011 / E114\\
Mo, Di, Mi, Fr: 8:30 -- 11:30~Uhr, Do: 13:30 -- 15:00~Uhr\\
\url{www.lmu.de/studentenkanzlei}


\subsection{Gleichstellungsbeauftragte}

\url{www.gleichstellungsbeauftragte.lmu.de}

\subsection{Frauenbeauftragte der Fakultäten}

Fakultät 16 (Mathe, Info, Statistik): \newline
\url{www.mathematik-informatik-statistik.uni-muenchen.de/fakultaet/beauftragte/index.html}\\

Fakultät 17 (Physik, Meteorologie)\newline
\url{www.physik.uni-muenchen.de/fakultaet/einrichtungen/frauenbauftragte/index.html}

\subsection{Studieren mit Kind}

\ldots ist nicht unmöglich. Die Uni bietet z. B. diverse Betreuungsmöglichkeiten.

\begin{itemize}
	\item \url{studentenwerk-muenchen.de/studieren-mit-kind}
	\item \url{www.lmu.de/studium/beratung/beratung_service/beratung_lmu/schwangere_kind}
\end{itemize}

\subsection{Studieren mit Behinderung}

Solltest du aufgrund einer Behinderung, wie einer fachärztlich festgestellten Legasthenie, mehr Zeit, spezielle Hilfsmittel oder einen eigenen Raum für Klausuren benötigen, so kannst du beim Prüfungsamt einen Nachteilsausgleich beantragen.

\begin{itemize}
	\item \url{studentenwerk-muenchen.de/studieren-mit-behinderung}
	\item \url{www.lmu.de/barrierefrei}
\end{itemize}



\subsection{Kirchliche Beratung}
Die christlichen Hochschulgemeinden bieten neben ihrem konfessionellen Angebot auch überkonfessionelle psychologische Beratung und Aktivitäten (Ausflüge, Spieleabende).

\paragraph{Katholische Hochschulgemeinde (KHG)}\hfill\\
Leopoldstr. 11, Raum 205\\
\url{www.khg.lmu.de}

\paragraph{Evangelische Hochschulgemeinde (ESG)}\hfill\\
Friedrichstr. 25\\
\url{www.esg.lmu.de}



\subsection{Psychosoziale Beratung}

Wenn du das Gefühl hast, die Kontrolle zu verlieren, oder nicht nur fachlich mit
dem Studium und den Menschen um dich herum zurecht kommst:\\
\url{www.studentenwerk-muenchen.de/beratungsnetzwerk/psychosoziale-und-psychotherapeutische-beratung/}


\subsection{Sonstige Beratung des Studentenwerks}
Helene-Mayer-Ring 9 (U3 Olympiazentrum)\\
\url{studentenwerk-muenchen.de/beratungsnetzwerk}

\begin{itemize}
	\item Allgemeine und Soziale Beratung
	\item Psychotherapeutische und Psychosoziale Beratungsstelle
	\item Studienkreditberatung
	\item Rechtsberatung
	\item Wohnungsberatung/ Privatzimmervermittlung
	\item Beratungsstelle „Sexuelle Belästigung, Diskriminierung und Gewalt“
	\item Beratung für ausländische Studierende
\end{itemize}

