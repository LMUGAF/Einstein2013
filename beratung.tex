\chapter{Hilfe und Beratung}

\section{Erste Hilfe: GAF}

Wir kennen nicht immer die Lösung, wissen dafür aber meistens, wer sie
kennt. Wir haben gute Kontakte zu vielen Institutionen und Personen an
dieser Uni. Wenn du uns einfach mal besuchen willst, bist du herzlich
willkommen. (Kontakt siehe Kapitel \ref{gafKontakt}, S. \pageref{gafKontakt})

\section{Probleme mit Lehrveranstaltungen oder Lehrpersonal}

Die offizielle Ansprechstelle hierbei ist der Studiendekan deiner
Fakultät. Er ist für die Qualität der Lehre verantwortlich. In jedem
Fall ist der sinnvollste Weg zu einer Lösung erst einmal das direkte
Gespräch mit dem Dozenten. Erst wenn ihr das Gefühl habt, ein Problem
lässt sich nicht anders lösen, bittet euren Studiendekan um
Hilfe. Oder fragt uns von der GAF.

\subsection*{Studiendekane Fakultät 16}
Prof. Werner Bley (Mathematik)\subjectList{\subjectM\subjectW}\\ %TODO
Prof. Hans Jürgen Ohlbach (Informatik)\subjectList{\subjectI\subjectMI}\\
Prof. Thomas Augustin (Statistik)\subjectList{\subjectS}%TODO

\subsection*{Studiendekan Fakultät 17}
Prof. Dr. Jochen Weller \subjectList{\subjectP}

\section{Webforen und Kommilitonen}

In den Foren kannst du dich mit deinen Kommilitonen
(und teils auch mit Lehrpersonal) austauschen. Wenn du Fragen direkt zu den
Übungen oder der Vorlesung hast, kannst du dich auch einfach an die
Übungsleiter der jeweiligen Vorlesung wenden. Keine Sorge, die beißen nur
selten.\\

%TODO genaueres zum IRC, evtl. mit Anleitung?
Wenn du im IRC unterwegs bist, findest du unter \#gaf und \#informatik.lmu
auch immer andere Studenten aus deinem Fach.

\begin{urlList}
	\urlItem{http://die-informatiker.net}
	\urlItem{http://die-physiker.org}
	\urlItem{http://die-mathematiker.net}
\end{urlList}

\section{Ansprechpartner}

Alle nachfolgenden Personen sind sehr umgängliche Menschen, mit denen
man bestens reden kann. Wie die meisten Professoren beißen sie nicht,
wenn man etwas zu beanstanden hat.

\subsection*{Mathematik (B.Sc., LA Gymnasium)\subjectList{\subjectM}}
PD Dr. Heribert Zenk (\mail{Heribert.Zenk@mathematik.uni-muenchen.de})\\
Theresienstr. 39, B333, Telefon: 089 / 2180 \emd{} 4460

PD Dr. Hartmut Weiß (\mail{hartmut.weiss@mathematik.uni-muenchen.de})\\
Theresienstr. 39, B317, Telefon: 089 / 2180 \emd{} 4680\\
Sprechstunde: Do, 15:00\emd{}16:00~Uhr

\subsection*{Wirtschaftsmathematik (B.Sc.)\subjectList{\subjectW}}
Prof. Dr. Gregor Svindland (\mail{studienberatung.wirtschaftsmathematik@math.lmu.de})\\
Theresienstr. 39, B231, Telefon: 809 / 2180 \emd{} 4628

\subsection*{Mathematik (LA Grund-, Haupt-, und Realschule)\subjectList{\subjectM}}
Dr. Erwin Schörner (\mail{schoerner@lmu.de})\\
Theresienstr. 39, B237, Telefon: 089 / 2180 \emd{} 4498

\subsection*{Mathematik (Fachdidaktik und Didaktik)\subjectList{\subjectM}}
Prof. Dr. Hedwig Gasteiger (\mail{gasteiger@math.lmu.de})\\ %TODO
Theresienstr. 39, B215, Telefon: 089 / 2180 \emd{} 4631

\subsection*{Informatik (B.Sc.)\subjectList{\subjectI}}
Dr. Reinhold Letz (\mail{reinhold.letz@lmu.de})\\
Oettingenstr. 67, E001, Telefon: 089 / 2180 \emd{} 9693\\
Sprechstunde: Di \& Mi 13:00\emd{}14:00~Uhr und nach Vereinbarung

\subsection*{Informatik (LA)\subjectList{\subjectI}}
Prof. Martin Hofmann, Ph.D. (\mail{lehramt@ifi.lmu.de})\\
Oettingenstr. 67, L107, Telefon: 089 / 2180 \emd{} 9341

\subsection*{Medieninformatik (B.Sc.)\subjectList{\subjectMI}}
Max Maurer / Simon Stusak (\mail{studentenbetreuer@medien.ifi.lmu.de})\\
Amalienstr. 17, 505, Telefon: 089 / 2180 \emd{} 4654

\subsection*{Physik (B.Sc.)\subjectList{\subjectP}}
Michael Rebhan (\mail{Michael.Rebhan@physik.uni-muenchen.de})\\
Schellingstr. 4, H417, Telefon: 089 / 2180 \emd{} 5033

\subsection*{Physik plus Meteorologie (B.Sc.)\subjectList{\subjectP}}
Dipl. Met. Heinz Lösslein (\mail{loesslein@lmu.de})\\
Theresienstr. 37, A208, Telefon: 089 / 2180 \emd{} 4217

\subsection*{Physik (LA)\subjectList{\subjectP}}
Prof. Dr. Raimund Girwidz (\mail{girwidz@physik.uni-muenchen.de})\\
Theresienstr. 37, A012, Telefon: 089 / 2180 \emd{} 2020


\section{Prüfungsamt}
Die Prüfungsämter sind für alle Prüfungsangelegenheiten zuständig,
also unter anderem für deine Noten, deine Praktika, deine Notenübersichten und
Abschlusszeugnisse. Sie sind bei der Fakultät zu finden, zu der
dein Studienfach gehört. Eine Zuordnung der Prüfungsämter zu den
einzelnen Studiengängen/-fächern findest du auf der Übersichtsseite
Studiengänge A--Z am unteren Ende der jeweiligen
Studiengangsinformationen.

\begin{urlList}
	\urlItem{http://www.lmu.de/pruefungsaemter}
\end{urlList}

\section{Studentenkanzlei}

Die Studentenkanzlei muss wegen gewissen formalen Belangen
gelegentlich besucht werden. Der Besuch dieses kafkaesken Molochs ist
oft mit großen Wartezeiten und Unbill verbunden. Es hilft, hartnäckig
zu bleiben und notfalls mehrfach zu kommen, bis du den richtigen
Sachbearbeiter triffst. Nicht umgehen lässt sich ein Besuch bei:

\begin{itemize}
\item Beantragen von Beurlaubungen (Krankheit, Ausland, Kinder\ldots)
\item Fragen zur Studienplatzvergabe/\,Immatrikulation (Anerkennung von Hochschulzugangsberechtigungen, nachträgliches Einschreiben, Verlust der Immatrikulationsbescheinigung)
\item Studienfachwechsel, zusätzliche Einschreibung für ein Doppelstudium
\item Bescheinigungen für die Krankenkasse und Rente, Quittungen für Studienbeiträge
\end{itemize}

HGB, E011\,/\,E114\\
Mo, Di, Mi, Fr: 8:30 -- 11:30~Uhr, Do: 13:30 -- 15:00~Uhr

\begin{urlList}
	\urlItem{http://www.lmu.de/studentenkanzlei}
\end{urlList}

%\section{Gleichstellungsbeauftragte}
%
%\begin{urlList}
%	\urlItem{www.gleichstellungsbeauftragte.lmu.de}
%\end{urlList}

\section{Die Frauenbeauftragten}
Weitere Anlaufstellen im Uni-Alltag vor allem bei Fragen und Problemen bezüglich Diskriminierungen und sexueller Belästigung im Wissenschaftsbetrieb sind die Frauenbeauftragten.
Das Aufgabengebiet der Frauenbeauftragten ist vielfältig und groß, darum hat zusätzlich zur Universitätsfrauenbeauftragten jede Fakultät eigene Frauenbeauftragte.

Alle Studierende können an den Weiterbildungsprogramm LMU-PLUS, welches durch das Büro der Frauenbeauftragten organisiert und aus Studienersatzmitteln finanziert wird, teilnehmen. Ausschließlich zur Förderung von Frauen ist das LMUMentoring und die Beratung zur finanziellen Förderung von Nachwuchswissenschaftlerinnen gedacht.

\begin{urlList}
        \urlItem{http://www.frauenbeauftragte.uni-muenchen.de/weiterbildung/plus/index.html}
	\urlItem{http://www.mathematik-informatik-statistik.uni-muenchen.de/fakultaet/beauftragte/index.html}
	\urlItem{http://www.physik.uni-muenchen.de/fakultaet/einrichtungen/frauenbauftragte/index.html}
\end{urlList}

\section{Studieren mit Kind}

Auch für Eltern ist Studieren nicht unmöglich. Die Uni bietet diverse Beratungs"~ und Betreuungsmöglichkeiten.

\begin{urlList}
	\urlItem{http://studentenwerk-muenchen.de/studieren-mit-kind}
	\urlItem{http://www.lmu.de/studium/beratung/beratung_service/beratung_lmu/schwangere_kind}
\end{urlList}

\section{Studieren mit Behinderung}

Solltest du aufgrund einer Behinderung mehr Zeit, spezielle Hilfsmittel oder einen eigenen Raum für Klausuren benötigen, so kannst du beim Prüfungsamt einen Nachteilsausgleich beantragen.

\begin{urlList}
	\urlItem{http://studentenwerk-muenchen.de/studieren-mit-behinderung}
	\urlItem{http://www.lmu.de/barrierefrei}
\end{urlList}



\section{Kirchliche Beratung}
Die christlichen Hochschulgemeinden bieten neben ihrem konfessionellen Angebot auch überkonfessionelle und psychologische Beratung und Aktivitäten, wie Ausflüge, Workshops, Spieleabende\ldots

\begin{urlList}
	\urlItem{http://www.khg.lmu.de}
	\urlItem{http://www.esg.lmu.de}
\end{urlList}

\section{Psychosoziale Beratung}

Wenn du das Gefühl hast, die Kontrolle zu verlieren, oder nicht mehr mit
dem Studium und/oder den Menschen um dich herum zurecht kommst, wende dich an die Psychosoziale Beratung des Studentenwerks.

\begin{urlList}
	\urlItem{http://www.studentenwerk-muenchen.de/beratungsnetzwerk/psychosoziale-und-psychotherapeutische-beratung/}
\end{urlList}

\section{Nightline München}

Die Nightline München ist ein Zuhörtelefon von Studika für Studika,
das abends und nachts zu erreichen ist. Am Telefon sitzen ehrenamtlich
tätige Studika die dir mit einem offenen Ohr beistehen.

\begin{urlList}
	\urlItem{http://www.nightline.mhn.de/}
\end{urlList}


\section{Sonstige Beratung des Studentenwerks}
Helene-Mayer-Ring 9 (U3 Olympiazentrum)

\begin{itemize}
	\item Allgemeine und Soziale Beratung
	\item Psychotherapeutische Beratungsstelle
	\item Studienkreditberatung
	\item Rechtsberatung
	\item Wohnungsberatung/\,Privatzimmervermittlung
	\item Beratungsstelle „Sexuelle Belästigung, Diskriminierung und Gewalt“
	\item Beratung für ausländische Studierende
\end{itemize}

\begin{urlList}
	\urlItem{http://studentenwerk-muenchen.de/beratungsnetzwerk}
\end{urlList}

\section{Student und Arbeitsmarkt}

Der Career Service der Universität bietet dir eine Stellen- und Praktikavermittlung, Kompetenztrainings, ein Mentoringprogramm, verschiedene Recruitung Events und einiges mehr. Einen Überblick verschaffst du dir am besten online oder du besuchst sie in Ludwigstraße 27 / I. Stock am Montag, Diesntag, Donnerst und Freitag zwischen 10 und 12~Uhr.

\begin{urlList}
	\urlItem{http://www.s-a.lmu.de}
\end{urlList}

