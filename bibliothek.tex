\chapter{Bibliotheken}

\section{Bücher}

Bei Verständnisschwierigkeiten des Stoffes hilft es nicht nur deine Kommilitonika
um Rat zu fragen, sondern auch Bücher zu lesen. Die Bibliothek
hat einen großen Bestand an Büchern, die du zum Teil auch ausleihen kannst.
In der Regel sind die von den Professoren empfohlenen Bücher
mehrfach vorhanden, allerdings oft schnell vergriffen. Falls ein
von dir benötigtes Buch nicht vorhanden sein sollte:
Anschaffungswünsche werden innerhalb von etwa einem Monat erfüllt. Außerdem gibt es
online eine große Auswahl an Ebooks, die dir kostenlos zur Verfügung stehen, siehe
auch \ref{sec:online}.

\section{Recherche im OPAC}
	\begin{urlList}
		\urlItem{http://opacplus.ub.uni-muenchen.de}
		\urlItem{http://www.ub.lmu.de}
	\end{urlList}

\section{Verhalten in der Bibliothek}
Verboten sind je nach Bibliothek: Rauchen, Essen, Getränke (außer Wasser in durchsichtigen Flaschen), Mäntel, Jacken, Taschen, Handyklingeln, Unterhalten.\\
Die Verbote variieren je nach Einrichtung relativ stark. Bitte informiere dich vorher online darüber.
Die Bibliotheken werden auch gerne einfach als ruhiger Ort zum Lernen genutzt.

\textbf{Bitte verhalte dich leise!
Deine lernenden Kommillitonen werden es dir danken.}

\section{Ausleihe}

Bücher in der Zentralen Lehrbuchsammlung (ehemals
Studentenbibliothek) und anderen Fachbibliotheken sind fast alle
ausleihbar. Bei Präsenzbibliotheken ist die Ausleihe nur über das
Wochenende möglich.

Beachte die Ausleihfristen (Mahngebühren variieren je nach Bibliothek!). 
Verlängerungen sind unter \ref{verlängerung}
möglich, vorausgesetzt, du hast noch keine ausstehenden Mahngebühren.

Gebühren kannst du an den Automaten in der Theresienstraße sowie
im Hauptgebäude begleichen.

\begin{urlList}
	\urlItem{http://opacplus.ub.uni-muenchen.de}[verlängerung]
\end{urlList}

\section{Die wichtigsten Bibliotheken für dich}

\subsection*{Fachbibliothek für Mathematik, Physik und Meteorologie\subjectList{\subjectM\subjectW\subjectP}}
Theresienstraße 37 (1. Stock)\\
Öffnungszeiten: Mo -- Fr 8:00 -- 22:00~Uhr, Sa 9:00 -- 18:00~Uhr\\
Buchscanner, Kopierer/Scanner mit Kartenzahlung, Basisbibliothek aller
Studenten der Fakultäten 16/17, Diskussionsräume für Gruppenarbeit.
Zwei große Lese"~ und Arbeitssäale.

\subsection*{Fachbibliothek Englischer Garten\subjectList{\subjectI\subjectMI}}
Oettingenstraße 67 (Haupteingang, Erdgeschoss)\\
Öffnungszeiten: Mo -- Fr 8:00 -- 22:00~Uhr, Sa -- So 9:00 -- 18:00~Uhr\\
Präsenzbibliothek Informatik, Münz- und Kartenkopierer, Ausleihe von maximal fünf Büchern, nur für Informatik-Studenten und nur über das Wochenende (Fr, 11:00 -- Mo, 12:00~Uhr).
%Am Sonntag ist keine Ausleihe oder Fachauskunft möglich. Außerdem musst du, um sonntags in die Oettingenstraße zu kommen,
%am Haupteingang den Wachdienst auf die aufmerksam machen (zum Beispiel ans Fenster klopfen) damit er dir aufmachen kann.

\subsection*{Zentralbibliothek der LMU}
Geschwister-Scholl-Platz 1 (Hauptgebäude Südtrakt)\\
Öffnungszeiten: Mo -- Fr 9:00 -- 22:00~Uhr\\
Serviceschalter: Mo -- Fr 9:00 -- 18:00~Uhr\\
Anlaufstelle bei verlorener Bibliotheksausweis und Abholung von Büchern aus dem Zentralbestand.

\subsection*{Bibliothek der TUM in der Innenstadt}
Arcisstraße 21d\\
Öffnungszeiten: Mo -- Fr 8:00 -- 24:00~Uhr, Sa, So und Feiertage 09:00 -- 22:00~Uhr\\
Für alle Studenten frei zum Lernen, einen TUM-Bibliotheksausweis erhältst du gegen Vorlage des Studienausweises an der Information.

\begin{urlList}
	\urlItem{http://www.ub.tum.de}
\end{urlList}

\subsection*{Bayerische Staatsbibliothek (Stabi)}
Ludwigstraße 16\\
Öffnungszeiten Ortsleihe: Mo -- Fr 09:00 -- 19:00~Uhr\\
Öffnungszeiten Lesesaal: täglich (auch Sonntags!) 8:00 -- 24:00~Uhr\\
Gewaltiger Bestand (Noten, Zeitschriften, Antikes\ldots), Bücher
müssen online bestellt werden, Ausleihe mit deiner LMU-Bibliotheksausweis. Wer
einen Arbeitsplatz ergattern möchte, sollte früh da sein; der
Ansturm an Lernwilligen ist immens.
Es herrschen jedoch auch relativ strenge Benutzungsbedingungen, so wird der Wachmann schonmal nervös, wenn man in größeren Gruppen dort aufschlägt.
Zum Trinken darf nur Wasser in durchsichtigen Flaschen mitgebracht werden.

\begin{urlList}
	\urlItem{http://bsb-muenchen.de}
\end{urlList}

\subsection*{Bibliothek des Deutschen Museums}
Auf der Museumsinsel\\
Öffnungszeiten: täglich (auch Sonntags!) 9:00 -- 17:00~Uhr\\
Große Auswahl an technischen und naturwissenschaftlichen Werken, Präsenzbibliothek, schönes Gebäude.

\begin{urlList}
	\urlItem{http://deutsches-museum.de/bibliothek}
\end{urlList}

\subsection*{Münchener Stadtbibliothek (Hauptstelle am Gasteig)}
Rosenheimer Straße 5\\
Öffnungszeiten: Mo -- Fr 10:00 -- 19:00~Uhr und Sa 11:00 -- 16:00~Uhr\\
Rückgabe täglich 7:00 -- 23:00~Uhr\\
Niederlassungen über die ganze Stadt verteilt, Ausleihe für Studenten 10~€ pro Jahr.

\begin{urlList}
	\urlItem{http://muenchner-stadtbibliothek.de}
\end{urlList}
