\chapter{Bibliotheken}

\section{Bücher}

Bei Verständnisschwierigkeiten des Stoffes hilft es -- neben
Kommilitonen um Rat zu fragen -- Bücher zu lesen.  Die Bibliothek
hat einen großen Bestand an Büchern, welche teilweise auch ausleihbar
sind. In der Regel sind die von den Professoren empfohlenen Bücher
mehrfach vorhanden, allerdings oft auch schnell vergriffen. Falls ein
von dir benötigtes Buch nicht vorhanden sein sollte:
Anschaffungswünsche werden innerhalb von etwa einem Monat erfüllt.

\section{Recherche im OPAC}
	\begin{urlList}
		\urlItem{http://opacplus.ub.uni-muenchen.de}
		\urlItem{http://www.ub.lmu.de}
	\end{urlList}

\section{Verhalten in der Bibliothek}
Verboten sind je nach Bibliothek: Rauchen, Essen, Getränke (außer Wasser in Plastikflaschen), Mäntel, Jacken, Taschen, Handyklingeln, Unterhalten.\\
Die Verbote variieren je nach Einrichtung relativ stark. Bitte informiere dich vorher online darüber.
Die Bibliotheken werden auch gerne einfach als ruhiger Ort zum Lernen genutzt.

\textbf{Bitte verhalte dich leise!
Deine lernenden Kommillitonen werden es dir danken.}

\section{Ausleihe}

Bücher in der Zentralen Lehrbuchsammlung (ehemals
Studentenbibliothek) und anderen Fachbibliotheken sind fast alle
ausleihbar. Bei Präsenzbibliotheken ist die Ausleihe nur über das
Wochenende möglich.

Beachte die Ausleihfristen (Mahngebühren variieren je nach Bibliothek!). 
Verlängerungen sind unter \ref{verlängerung}
möglich, vorausgesetzt, du hast noch keine ausstehenden Mahngebühren.

Gebühren kannst du an den Automaten in der Theresienstraße sowie
im Hauptgebäude begleichen.

\begin{urlList}
	\urlItem{http://opacplus.ub.uni-muenchen.de}{verlängerung}
\end{urlList}

\section{Die wichtigsten Bibliotheken für dich}

\subsection*{Fachbibliothek für Mathematik, Physik und Meteorologie}
Theresienstr. 37 (1. Stock)\\
Öffnungszeiten: Mo -- Fr 8:00 -- 22:00~Uhr, Sa 9:00 -- 18:00~Uhr\\
Buchscanner, Kopierer/Scanner mit Kartenzahlung, Basisbibliothek aller
Studenten der Fakultäten 16/17, Diskussionsräume für Gruppenarbeit.
Zwei große Lese"~ und Arbeitssäale.

\subsection*{Fachbibliothek Englischer Garten}
Oettingenstr. 67 (Haupteingang, Erdgeschoss)\\
Öffnungszeiten: Mo -- Fr 8:00 -- 22:00~Uhr und Sa 9:00 -- 18:00~Uhr\\
Präsenzbibliothek Informatik, Münz- und Kartenkopierer, Ausleihe von max. fünf Büchern, nur für Info-Studenten und nur über das Wochenende (Fr, 11:00 -- Mo, 12:00~Uhr).

\subsection*{Zentralbibliothek der LMU}
Geschwister-Scholl-Platz 1 (Hauptgebäude Südtrakt)\\
Öffnungszeiten: Mo -- Fr 9:00 -- 22:00~Uhr, Fr 9:00 -- 17:00~Uhr\\
Serviceschalter: Mo -- Fr 9:00 -- 20:00~Uhr\\
Anlaufstelle bei verlorener Bib-Karte und Abholung von Büchern aus dem Zentralbestand.

\subsection*{Bibliothek der TUM in der Innenstadt}
Arcisstr. 21d\\
Öffnungszeiten: Mo -- Fr 8:00 -- 24:00~Uhr, Sa, So und Feiertage 10:00 -- 22:00~Uhr\\
Für alle Studenten frei zum Lernen, einen TUM-Bibliotheksausweis erhältst du gegen Vorlage des Studienausweises an der Information.

\begin{urlList}
	\urlItem{http://www.ub.tum.de}
\end{urlList}

\subsection*{Bayerische Staatsbibliothek (Stabi)}
Ludwigstr. 16\\
Öffnungszeiten Ortsleihe: Mo -- Fr 10:00 -- 19:00~Uhr\\
Öffnungszeiten Lesesaal: täglich (auch Sonntags!) 8:00 -- 24:00~Uhr\\
Gewaltiger Bestand (Noten, Zeitschriften, Antikes, \ldots), Bücher
müssen online bestellt werden, Ausleihe mit deiner LMU-Bib-Karte. Wer
einen Arbeitsplatz ergattern möchte, sollte früh da sein; der
Ansturm an Lernwilligen ist immens.

\begin{urlList}
	\urlItem{http://bsb-muenchen.de}
\end{urlList}

\subsection*{Bibliothek des Deutschen Museums}
Auf der Museumsinsel\\
Öffnungszeiten: täglich (auch Sonntags!) 9:00 -- 17:00~Uhr\\
Große Auswahl an technischen und naturwissenschaftlichen Werken, Präsenzbibliothek, schönes Gebäude.

\begin{urlList}
	\urlItem{http://deutsches-museum.de/bibliothek}
\end{urlList}

\subsection*{Münchener Stadtbibliothek (Hauptstelle am Gasteig)}
Rosenheimer Str. 5\\
Öffnungszeiten: Mo -- Fr 10:00 -- 19:00~Uhr und Sa 11:00 -- 16:00~Uhr\\
Rückgabe täglich 7:00 -- 23:00~Uhr\\
Niederlassungen über die ganze Stadt verteilt, Ausleihe für Studenten 10~€ pro Jahr.

\begin{urlList}
	\urlItem{http://muenchner-stadtbibliothek.de}
\end{urlList}
