
\section{Ausland und Praktika}

Auslandssemester oder -praktika machen sich immer gut im Lebenslauf,
und sind nebenbei bleibende Erinnerungen, von denen viele von uns mehr
profitiert haben als von der ein oder anderen Vorlesung.  Und falls du
dich f�r ein Thema besonders interessierst, bieten auch viele
Hochschulen im Ausland die M�glichkeit, eine Abschlussarbeit bei ihnen
zu verfassen bzw. verfassen zu lassen (\url{www.jura.uni-bayreuth.de}).

Hierbei wirst du Uni-intern vom \emph{Referat Internationale Angelegenheiten} und dem Career Center \emph{Student und Arbeitsmarkt} unterst�tzt, aber auch von Hochschulgruppen wie AISEC oder IAESTE (vom DAAD gef�rdert).

\subsection{Auslandsstudium}

Die LMU verf�gt �ber eine Reihe von Partnerhochschule in aller
Welt. Der Austausch ist hier einfacher (Formalien, Anerkennung von
ECTS) und du wirst von den Studiengeb�hren befreit. F�r die
Partnerhochschulen kann man sich nur ein Mal im Jahr bewerben, also am
besten fr�hzeitig �ber Fristen informieren und bewerben.
Ein Jahr vor der Abreise ist manchmal schon zu sp�t, um sich bei
allen Organisationen (insb. DAAD) zu bewerben.
Es ist aber auch m�glich, sich selbst einen Austausch an einer anderen
Hochschule zu organisieren.

Falls du im Ausland erworbene ECTS an der LMU anerkennen lassen
m�chtest, solltest du dies im Vorfeld mit dem Studiengangskoordinator
abkl�ren.

Austauschabkommen und -vertr�ge, sowie Erfahrungsberichte findest du unter:\\
\url{https://www.moveon.verwaltung.uni-muenchen.de/move/moveonline/exchanges/search.php?_language=de}

\subsection{Praktikum im In- und Ausland}

Neben Jobb�rsen gibt es auch Datenbanken wie die des DAAD
(\url{eu-community.daad.de/index.php?id=38}) mit
Praktikums-Erfahrungsberichten. So kann man sich im Vorfeld schon
einen groben �berblick �ber das jeweilige Praktikum machen.

\subsection{Finanzierung}

Dies ist nur eine Auswahl der M�glichkeiten. F�r bestimmte L�nder und
Vorhaben gibt es auch noch spezielle finanzielle Unterst�tzungen. Die
Vorlaufzeit betr�gt 3--18 Monate.

\begin{itemize}
\item Auslands-BAF�G: staatliche finanzielle F�rderung (nicht zur�ckzuzahlen) f�r ein Studium oder Praktikum im Ausland. Hierbei sind auch viele f�rderungsberechtigt, die kein regul�res BAF�G erhalten, also auf jeden Fall bewerben!
\item ERASMUS: ein Stipendiumprogramm f�r ein 3- bis 12-monatiges Studium oder Praktikum im europ�ischen Ausland.
\item DAAD und PROSA LMU: Stipendien f�r Studium, Praktikum, Sprachkurse und Kurzprogramme im Ausland.
\end{itemize}

Referat Internationale Angelegenheiten\\
HGB, G013, Zugang �ber G011\\
Mo--Fr: 9:00--11.30~Uhr, Mi: 13:00--15:00~Uhr\\
\url{www.lmu.de/international/auslandsstudium}\\


Student und Arbeitsmarkt\\
HGB, G206\\
Mo, Di, Do und Fr: 10:00--12:00~Uhr\\
\url{www.s-a.lmu.de}

