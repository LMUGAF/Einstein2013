
\section{Ausland und Praktika}

Auslandssemester oder -praktika machen sich immer gut im Lebenslauf,
und sind nebenbei bleibende Erinnerungen, von denen viele von uns mehr
profitiert haben als von der ein oder anderen Vorlesung.  Und falls du
dich für ein Thema besonders interessierst, bieten auch viele
Hochschulen im Ausland die Möglichkeit, eine Abschlussarbeit bei ihnen
zu verfassen bzw. verfassen zu lassen (\url{www.jura.uni-bayreuth.de}).

Hierbei wirst du Uni-intern vom \emph{Referat Internationale Angelegenheiten} und dem Career Center \emph{Student und Arbeitsmarkt} unterstützt, aber auch von Hochschulgruppen wie AISEC oder IAESTE (vom DAAD gefördert).

\subsection{Auslandsstudium}

Die LMU verfügt über eine Reihe von Partnerhochschule in aller
Welt. Der Austausch ist hier einfacher (Formalien, Anerkennung von
ECTS) und du wirst von den Studiengebühren befreit. Für die
Partnerhochschulen kann man sich nur ein Mal im Jahr bewerben, also am
besten frühzeitig über Fristen informieren und bewerben.
Ein Jahr vor der Abreise ist manchmal schon zu spät, um sich bei
allen Organisationen (insb. DAAD) zu bewerben.
Es ist aber auch möglich, sich selbst einen Austausch an einer anderen
Hochschule zu organisieren.

Falls du im Ausland erworbene ECTS an der LMU anerkennen lassen
möchtest, solltest du dies im Vorfeld mit dem Studiengangskoordinator
abklären.

Austauschabkommen und -verträge, sowie Erfahrungsberichte findest du unter:\\
\url{https://www.moveon.verwaltung.uni-muenchen.de/move/moveonline/exchanges/search.php?_language=de}

\subsection{Praktikum im In- und Ausland}

Neben Jobbörsen gibt es auch Datenbanken wie die des DAAD
(\url{eu-community.daad.de/index.php?id=38}) mit
Praktikums-Erfahrungsberichten. So kann man sich im Vorfeld schon
einen groben Überblick über das jeweilige Praktikum machen.

\subsection{Finanzierung}

Dies ist nur eine Auswahl der Möglichkeiten. Für bestimmte Länder und
Vorhaben gibt es auch noch spezielle finanzielle Unterstützungen. Die
Vorlaufzeit beträgt 3--18 Monate.

\begin{itemize}
\item Auslands-BAFöG: staatliche finanzielle Förderung (nicht zurückzuzahlen) für ein Studium oder Praktikum im Ausland. Hierbei sind auch viele förderungsberechtigt, die kein reguläres BAFöG erhalten, also auf jeden Fall bewerben!
\item ERASMUS: ein Stipendiumprogramm für ein 3- bis 12-monatiges Studium oder Praktikum im europäischen Ausland.
\item DAAD und PROSA LMU: Stipendien für Studium, Praktikum, Sprachkurse und Kurzprogramme im Ausland.
\end{itemize}

Referat Internationale Angelegenheiten\\
HGB, G013, Zugang über G011\\
Mo--Fr: 9:00--11.30~Uhr, Mi: 13:00--15:00~Uhr\\
\url{www.lmu.de/international/auslandsstudium}\\


Student und Arbeitsmarkt\\
HGB, G206\\
Mo, Di, Do und Fr: 10:00--12:00~Uhr\\
\url{www.s-a.lmu.de}

