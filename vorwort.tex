%\section{Über dieses Heft}
%
%Bei der ganzen Informationsflut, die in der ersten Unizeit auf dich
%einstürzt, hoffen wir dir mit unserem \emph{Ersti-Einstein} einen
%kleinen Ratgeber an die Hand zu geben.  Der Ersti-Einstein bündelt
%Wichtiges, erklärt dir Nichtoffensichtliches, und versucht bei vielen
%Problemen zumindest erste Lösungsansätze zu bieten.
%
%Da wir nicht mehr alle Probleme, die ein Ersti hat, nachvollziehen
%können, und jedes Jahr neue Probleme gefunden werden, würden wir uns
%freuen, wenn du uns fehlende Informationen unter
%\mail{einstein@fs.lmu.de} mitteilst.

\section{Don't Panic!}

Wenn du dieses Heft in der Hand hältst, wirst du schon mit vielen neuen Informationen bombardiert worden sein und es wird noch viel mehr auf dich zukommen. Damit du trotzdem den Überblick behältst, haben wir hier alles Wichtige zusammengefasst, was du jetzt und vielleicht später einmal brauchen wirst. Falls es doch irgendwann ein Problem gibt, bei dem dir unser \emph{Ersti-Einstein} keinen Lösungsansatz bietet, dann komm in unserem Fachschaftszimmer vorbei, frag dein Tutorikon oder schreib uns eine Mail an \mail{gaf@fs.lmu.de}, damit wir es gleich in die nächste Ausgabe aufnehmen können.

Wie du dein Studium organisierst, ist dir von nun an selbst überlassen. Das wirft natürlich Fragen auf wie "Was muss ich? Was kann ich? Was sollte ich?! und vor allem "Was will ich?". Damit du all deine neu gefundenen oder altbekannten Ziele erreichen kannst und nicht im kafkaesken Moloch der Univerwaltung untergehst, haben wir versucht, das Nichtoffensichtliche aufzuschreiben. Und wenn man sich einmal daran gewöhnt hat, wie die Dinge an der Uni ablaufen, sieht die Welt wieder ganz anders aus.

In diesem Sinne: Nutze deine Zeit und wen etwas mal nicht so läuft wie geplant, mach das Beste draus!

Deine Gruppe Aktiver Fachschaftika
