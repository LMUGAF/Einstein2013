%\section{Über dieses Heft}
%
%Bei der ganzen Informationsflut, die in der ersten Unizeit auf dich
%einstürzt, hoffen wir dir mit unserem \emph{Ersti-Einstein} einen
%kleinen Ratgeber an die Hand zu geben.  Der Ersti-Einstein bündelt
%Wichtiges, erklärt dir Nichtoffensichtliches, und versucht bei vielen
%Problemen zumindest erste Lösungsansätze zu bieten.
%
%Da wir nicht mehr alle Probleme, die ein Ersti hat, nachvollziehen
%können, und jedes Jahr neue Probleme gefunden werden, würden wir uns
%freuen, wenn du uns fehlende Informationen unter
%\mail{einstein@fs.lmu.de} mitteilst.

\chapter{Don't Panic!}

Wenn du dieses Heft in der Hand hältst, wirst du schon mit vielen neuen Informationen bombardiert worden sein und es wird noch viel mehr auf dich zukommen. Damit du trotzdem den Überblick behältst, haben wir hier alles Wichtige zusammengefasst, was du jetzt und vielleicht später einmal brauchen wirst. Falls es doch irgendwann ein Problem gibt, bei dem dir unser \emph{Ersti-Einstein} keinen Lösungsansatz bietet, dann komme in unserem Fachschaftszimmer vorbei, frage dein Tutorikon oder schreibe uns eine Mail an \mail{gaf@fs.lmu.de}, so dass wir es gleich in die nächste Ausgabe aufnehmen können. 

Falls du dich jetzt fragst, was ein Tutorikon ist: -ikon (Singular) bzw. -ika (Plural) ist eine geschlechtsneutrale Personenbezeichnung, die aus dem Griechischen stammt. Auf die Verwendung dieser Endungen haben wir uns nach sehr langen und intensiven Genderdiskussionen geeinigt. Bei dieser Frage geht es darum, wie man Frauen und andere Gender auch sprachlich gleichwertig zur Geltung bringen kann, obwohl das Deutsche oft nur eine männliche Form kennt. Mit Tutorikon ist also dein Tutor oder deine Tutorin gemeint, welche dir alle wichtigen Orte an der Uni zeigen. Diese Form ist erst ungewohnt, macht aber Sinn, schließt alle ein und macht, wie du bald feststellen wirst, auch ein bisschen Spaß.

Wie du dein Studium organisierst, liegt nun ganz in deiner Hand. Das wirft natürlich Fragen auf wie ``Was muss ich? Was kann ich? Was sollte ich?!'' und vor allem ``Was will ich?''. Damit du all deine neu gefundenen oder altbekannten Ziele erreichen kannst, haben wir versucht, das Nichtoffensichtliche aufzuschreiben. Und wenn man sich einmal daran gewöhnt hat, wie die Dinge an der Uni ablaufen, ist alles ganz einfach.

In diesem Sinne: Nutze deine Zeit und wenn etwas mal nicht so läuft wie geplant, frag uns und mach das Beste draus!

Deine Gruppe Aktiver Fachschaftika
\vfill

%TODO Warum Statistik die nicht dabei ist, aber keine Meteo?
\begin{table*}
	\centering
	  Fächerkennzeichnungen
		\begin{tabular}{ l l c l l }
			\subjectI & Informatik &  & \subjectMI & Medieninformatik \\[1.5mm]
			\subjectM & Mathematik &  & \subjectW  & Wirtschaftsmathematik \\[1.5mm]
			\subjectS & Statistik  &  & \subjectP  & Physik / Astro / Meteo
		\end{tabular}
\end{table*}
